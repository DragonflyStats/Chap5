\documentclass[Main.tex]{subfiles}
\begin{document}
\subsection{Residual Plots}
A residual plot is a graph that shows the residuals on the vertical axis and the independent variable on the horizontal axis. If the points in a residual plot are randomly dispersed around the horizontal axis, a linear regression model is appropriate for the data; otherwise, a non-linear model is more appropriate.

Below the table on the left shows inputs and outputs from a simple linear regression analysis, and the chart on the right displays the residual (e) and independent variable (X) as a residual plot.

\begin{verbatim}
x &	60	& 70	& 80	& 85 &	95 \\ \hline
y &	70	& 65	& 70	& 95 &	85 \\ \hline
y.hat	65.411	71.849	78.288	81.507	87.945
e	4.589	-6.849	-8.288	13.493	-2.945
\end{verbatim}


The residual plot shows a fairly random pattern - the first residual is positive, the next two are negative, the fourth is positive, and the last residual is negative. This random pattern indicates that a linear model provides a decent fit to the data.

Below, the residual plots show three typical patterns. The first plot shows a random pattern, indicating a good fit for a linear model. The other plot patterns are non-random (U-shaped and inverted U), suggesting a better fit for a non-linear model.




\bibliography{DB-txfrbib}
\end{document}