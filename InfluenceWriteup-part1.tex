\documentclass[12pt, a4paper]{report}
\usepackage{epsfig}
\usepackage{subfigure}
%\usepackage{amscd}
\usepackage{amssymb}
\usepackage{amsbsy}
\usepackage{amsthm}
%\usepackage[dvips]{graphicx}
\usepackage{natbib}
\bibliographystyle{chicago}
\usepackage{vmargin}
\usepackage{index}
% left top textwidth textheight headheight
% headsep footheight footskip
\setmargins{3.0cm}{2.5cm}{15.5 cm}{22cm}{0.5cm}{0cm}{1cm}{1cm}
\renewcommand{\baselinestretch}{1.5}
\pagenumbering{arabic}
\theoremstyle{plain}
\newtheorem{theorem}{Theorem}[section]
\newtheorem{corollary}[theorem]{Corollary}
\newtheorem{ill}[theorem]{Example}
\newtheorem{lemma}[theorem]{Lemma}
\newtheorem{proposition}[theorem]{Proposition}
\newtheorem{conjecture}[theorem]{Conjecture}
\newtheorem{axiom}{Axiom}
\theoremstyle{definition}
\newtheorem{definition}{Definition}[section]
\newtheorem{notation}{Notation}
\theoremstyle{remark}
\newtheorem{remark}{Remark}[section]
\newtheorem{example}{Example}[section]
\renewcommand{\thenotation}{}
\renewcommand{\thetable}{\thesection.\arabic{table}}
\renewcommand{\thefigure}{\thesection.\arabic{figure}}
\title{Research notes: linear mixed effects models}
\author{ } \date{ }

\makeindex
\begin{document}
\author{Kevin O'Brien}
\title{October 2011 Version A}

%---------------------------------------------------------------------------%
% - 1. Model Diagnostics
% - 2. Zewotir's paper (including Haslett)
% - 3. Augmented GLMS
% - 4. Applying Diagnostics to MCS
% - 5. Extra Material
%---------------------------------------------------------------------------%


\addcontentsline{toc}{section}{Bibliography}

\chapter{Model Diagnostics}
%---------------------------------------------------------------------------%
%1.1 Introduction to Influence Analysis
%1.2 Extension of techniques to LME Models
%1.3 Residual Diagnostics
%1.4 Standardized and studentized residuals
%1.5 Covariance Parameters
%1.6 Case Deletion Diagnostics
%1.7 Influence Analysis
%1.8 Terminology for Case Deletion
%1.9 Cook's Distance (Classical Case)
%1.10 Cook's Distance (LME Case)
%1.11 Likelihood Distance
%1.12 Other Measures
%1.13 CPJ Paper
%1.14 Matrix Notation of Case Deletion
%1.15 CPJ's Three Propositions
%1.16 Other measures of Influence
\newpage
%--------------------------------------------------------------------Diagnostics%
\section{Diagnostics}

%http://www.artifex.org/~meiercl/R_statistics_guide.pdf
\subsection{Identifying outliers with a LME model object}

The process is slightly different than with standard LME model objects, since the \textbf{\emph{influence}}
function does not work on lme model objects. Given \textbf{\emph{mod.lme}}, we can use the plot function to
identify outliers.
%----------------------%
\subsection{Diagnostics for Random Effects}
Empirical best linear unbiased predictors EBLUPS provide the a useful way of diagnosing random effects.

EBLUPs are also known as ``shrinkage estimators" because they tend to be smaller than the estimated effects would be if they were computed by treating a random factor as if it was fixed (West etal )




\newpage
%---------------------------------------------------------------------------%
\section{Introduction}%1.1
In classical linear models model diagnostics have been become a required part of any statistical analysis, and the methods are commonly available in statistical packages and standard textbooks on applied regression. However it has been noted by several papers that model diagnostics do not often accompany LME model analyses.
Model diagnostic techniques determine whether or not the distributional assumptions are satisfied, and to assess the influence of unusual observations.

\subsection{Model Data Agreement} %1.1.1
\citet{schabenberger} describes the examination of model-data agreement as comprising several elements; residual analysis, goodness of fit, collinearity diagnostics and influence analysis.

\subsection{Influence Diagnostics: Basic Idea and Statistics} %1.1.2
%http://support.sas.com/documentation/cdl/en/statug/63033/HTML/default/viewer.htm#statug_mixed_sect024.htm

The general idea of quantifying the influence of one or more observations relies on computing parameter estimates based on all data points, removing the cases in question from the data, refitting the model, and computing statistics based on the change between full-data and reduced-data estimation. 



\subsection{Influence Analysis for LME Models} %1.1.3
The linear mixed effects model is a useful methodology for fitting a wide range of models. However, linear mixed effects models are known to be sensitive to outliers. \citet{CPJ} advises that identification of outliers is necessary before conclusions may be drawn from the fitted model.

Standard statistical packages concentrate on calculating and testing parameter estimates without considering the diagnostics of the model.The assessment of the effects of perturbations in data, on the outcome of the analysis, is known as statistical influence analysis. Influence analysis examines the robustness of the model. Influence analysis methodologies have been used extensively in classical linear models, and provided the basis for methodologies for use with LME models.
Computationally inexpensive diagnostics tools have been developed to examine the issue of influence \citep{Zewotir}.
Studentized residuals, error contrast matrices and the inverse of the response variance covariance matrix are regular components of these tools.

\subsection{Influence Statistics for LME models} %1.1.4
Influence statistics can be coarsely grouped by the aspect of estimation that is their primary target:
\begin{itemize}
\item overall measures compare changes in objective functions: (restricted) likelihood distance (Cook and Weisberg 1982, Ch. 5.2)
\item influence on parameter estimates: Cook's  (Cook 1977, 1979), MDFFITS (Belsley, Kuh, and Welsch 1980, p. 32)
\item influence on precision of estimates: CovRatio and CovTrace
\item influence on fitted and predicted values: PRESS residual, PRESS statistic (Allen 1974), DFFITS (Belsley, Kuh, and Welsch 1980, p. 15)
\item outlier properties: internally and externally studentized residuals, leverage
\end{itemize}
%---------------------------------------------------------------------------%


\subsection{What is Influence} %1.1.5

Broadly defined, influence is understood as the ability of a single or multiple data points, through their presence or absence in the data, to alter important aspects of the analysis, yield qualitatively different inferences, or violate assumptions of the statistical model. The goal of influence analysis is not primarily to mark data
points for deletion so that a better model fit can be achieved for the reduced data, although this might be a result of influence analysis \citep{schabenberger}.

%-------%
\subsection{Quantifying Influence}  %1.1.6

The basic procedure for quantifying influence is simple as follows:

\begin{itemize}
\item Fit the model to the data and obtain estimates of all parameters.
\item Remove one or more data points from the analysis and compute updated estimates of model parameters.
\item Based on full- and reduced-data estimates, contrast quantities of interest to determine how the absence of the observations changes the analysis.
\end{itemize}

\citet{cook86} introduces powerful tools for local-influence assessment and examining perturbations in the assumptions of a model. In particular the effect of local perturbations of parameters or observations are examined.


%---------------------------------------------------------------------------%
\newpage
\section{Extension of techniques to LME Models} %1.2

Model diagnostic techniques, well established for classical models, have since been adapted for use with linear mixed effects models.Diagnostic techniques for LME models are inevitably more difficult to implement, due to the increased complexity.

Beckman, Nachtsheim and Cook (1987) \citet{Beckman} applied the \index{local influence}local influence method of Cook (1986) to the analysis of the linear mixed model.

While the concept of influence analysis is straightforward, implementation in mixed models is more complex. Update formulae for fixed effects models are available only when the covariance parameters are assumed to be known.

If the global measure suggests that the points in $U$ are influential, the nature of that influence should be determined. In particular, the points in $U$ can affect the following

\begin{itemize}
\item the estimates of fixed effects,
\item the estimates of the precision of the fixed effects,
\item the estimates of the covariance parameters,
\item the estimates of the precision of the covariance parameters,
\item fitted and predicted values.
\end{itemize}


%---------------------------------------------------------------------------%
\newpage


\subsection{Case Deletion Diagnostics for Mixed Models}

\citet{Christiansen} notes the case deletion diagnostics techniques have not been applied to linear mixed effects models and seeks to develop methodologies in that respect.

\citet{Christiansen} develops these techniques in the context of REML

\subsection{Methods and Measures}
The key to making deletion diagnostics useable is the development of efficient computational formulas, allowing one to obtain the \index{case deletion diagnostics} case deletion diagnostics by making use of basic building blocks, computed only once for the full model.

\citet{Zewotir} lists several established methods of analyzing influence in LME models. These methods include \begin{itemize}
\item Cook's distance for LME models,
\item \index{likelihood distance} likelihood distance,
\item the variance (information) ration,
\item the \index{Cook-Weisberg statistic} Cook-Weisberg statistic,
\item the \index{Andrews-Prebigon statistic} Andrews-Prebigon statistic.
\end{itemize}


%---------------------------------------------------------------------------%
\newpage
\section{Influence analysis} %1.7

Likelihood based estimation methods, such as ML and REML, are sensitive to unusual observations. Influence diagnostics are formal techniques that assess the influence of observations on parameter estimates for $\beta$ and $\theta$. A common technique is to refit the model with an observation or group of observations omitted.

\citet{west} examines a group of methods that examine various aspects of influence diagnostics for LME models.
For overall influence, the most common approaches are the `likelihood distance' and the `restricted likelihood distance'.

\subsection{Cook's 1986 paper on Local Influence}%1.7.1
Cook 1986 introduced methods for local influence assessment. These methods provide a powerful tool for examining perturbations in the assumption of a model, particularly the effects of local perturbations of parameters of observations.

The local-influence approach to influence assessment is quitedifferent from the case deletion approach, comparisons are of
interest.



\subsection{Overall Influence}
An overall influence statistic measures the change in the objective function being minimized. For example, in
OLS regression, the residual sums of squares serves that purpose. In linear mixed models fit by
\index{maximum likelihood} maximum likelihood (ML) or \index{restricted maximum likelihood} restricted maximum likelihood (REML), an overall influence measure is the \index{likelihood distance} likelihood distance [Cook and Weisberg ].

%--------------------------------------------------------------------------%
\newpage
\section{Terminology for Case Deletion diagnostics} %1.8

\citet{preisser} describes two type of diagnostics. When the set consists of only one observation, the type is called
'observation-diagnostics'. For multiple observations, Preisser describes the diagnostics as 'cluster-deletion' diagnostics.


\newpage
\section{Iterative and non-iterative influence analysis} %1.13
\citet{schabenberger} highlights some of the issue regarding implementing mixed model diagnostics.

A measure of total influence requires updates of all model parameters.

however, this doesnt increase the procedures execution time by the same degree.
\subsection{Iterative Influence Analysis}

%----schabenberger page 8
For linear models, the implementation of influence analysis is straightforward.
However, for LME models, the process is more complex. Update formulas for the fixed effects are available only when the covariance parameters are assumed to be known. A measure of total influence requires updates of all model parameters.
This can only be achieved in general is by omitting observations, then refitting the model.

\citet{schabenberger} describes the choice between \index{iterative influence analysis} iterative influence analysis and \index{non-iterative influence analysis} non-iterative influence analysis.


%----------------------------------------------------------------------------------------%
\newpage
\section{The CPJ Paper}%1.13

\subsection{Case-Deletion results for Variance components}
\citet{CPJ} examines case deletion results for estimates of the variance components, proposing the use of one-step estimates of variance components for examining case influence. The method describes focuses on REML estimation, but can easily be adapted to ML or other methods.

This paper develops their global influences for the deletion of single observations in two steps: a one-step estimate for the REML (or ML) estimate of the variance components, and an ordinary case-deletion diagnostic for a weighted regression problem ( conditional on the estimated covariance matrix) for fixed effects.

% Lesaffre's approach accords with that proposed by Christensen et al when applied in a repeated measurement context, with a large sample size.

\subsection{CPJ Notation} %1.13.1

\[ \boldsymbol{C} = \boldsymbol{H}^{-1} = \left[
\begin{array}{cc}
c_{ii} & \boldsymbol{c}_{i}^{\prime}\\
\boldsymbol{c}_{i} &  \boldsymbol{C}_{[i]}
\end{array} \right]
\]

\citet{CPJ} noted the following identity:

\[ \boldsymbol{H}_{[i]}^{-1}  = \boldsymbol{C}_{[i]} - {1 \over c_{ii}}\boldsymbol{c}_{[i]}\boldsymbol{c}_{[i]}^{\prime} \]


\citet{CPJ} use the following as building blocks for case deletion statistics.
\begin{itemize}
\item $\breve{x}_i$
\item $\breve{z}_i$
\item $\breve{z}_ij$
\item $\breve{y}_i$
\item $p_ii$
\item $m_i$
\end{itemize}
All of these terms are a function of a row (or column) of $\boldsymbol{H}$ and $\boldsymbol{H}_{[i]}^{-1}$

%---------------------------------------------------------------------------%
\newpage
\section{Matrix Notation for Case Deletion} %1.14

\subsection{Case deletion notation} %1.14.1

For notational simplicity, $\boldsymbol{A}(i)$ denotes an $n \times m$ matrix $\boldsymbol{A}$ with the $i$-th row
removed, $a_i$ denotes the $i$-th row of $\boldsymbol{A}$, and $a_{ij}$ denotes the $(i, j)-$th element of $\boldsymbol{A}$.

\subsection{Partitioning Matrices} %1.14.2
Without loss of generality, matrices can be partitioned as if the $i-$th omitted observation is the first row; i.e. $i=1$.

%---------------------------------------------------------------------------%
\newpage
\section{CPJ's Three Propositions} %1.15
%-----------------------------%


\subsubsection{Proposition 1}

\[
\boldsymbol{V}^{-1} =
\left[ \begin{array}{cc}
\nu^{ii} & \lambda_{i}^{\prime}  \\
\lambda_{i} & \Lambda_{[i]}
\end{array}\right] \]


\[\boldsymbol{V}_{[i]}^{-1} = \boldsymbol{\Lambda}_{[i]} - { \lambda_{i} \lambda_{i} ^{\prime} \over \lambda_{i} } \]

%-----------------------------%
\subsection{Proposition 2}

\begin{itemize}
\item[(i)] $ \boldsymbol{X}_{[i]}^{T}\boldsymbol{V}^{-1}_{[i]}\boldsymbol{X}_{[i]}$ = $\boldsymbol{X}^{\prime}\boldsymbol{V}^{-1}\boldsymbol{X}$
\item[(ii)] = $(\boldsymbol{X}^{\prime}\boldsymbol{V}^{-1}\boldsymbol{Y})^{-1}$
\item[(iii)] $ \boldsymbol{X}_{[i]}^{T}\boldsymbol{V}^{-1}_{[i]}\boldsymbol{Y}_{[i]}$ = $\boldsymbol{X}^{\prime}\boldsymbol{V}^{-1}\boldsymbol{Y}$
\end{itemize}
%-----------------------------%
\subsection{Proposition 3}
This proposition is similar to the formula for the one-step Newtown Raphson estimate of the logistic regression coefficients given by Pregibon (1981) and discussed in Cook Weisberg.
%-------------------------------------------------------------------------------------------------Chapter 2	------------------------%
%-------------------------------------------------------------------------------------------------------------------------------------%
%-------------------------------------------------------------------------------------------------------------------------------------%

\chapter{Zewotir's Paper}

% 2.1 Efficient Updating Theorem
% 2.2 Zewotir Measures of Influence in LME Models (section 4 of paper)
% 2.3 Computation and Notation 
% 2.4 Measures 2
%2.5 Haslett Analysis

\section{Efficient Updating Theorem} %2.1
\citet{Zewotir} describes the basic theorem of efficient updating.
\begin{itemize}
\item \[ m_i = {1 \over c_{ii}}\]
%\item
%item
%\item
\end{itemize}
%-------------------------------------------------------------------------------------------------------------------------------------%
\section{Zewotir Measures of Influence in LME Models}%2.2
%Zewotir page 161
\citet{Zewotir} describes a number of approaches to model diagnostics, investigating each of the following;
\begin{itemize}
\item Variance components
\item Fixed effects parameters
\item Prediction of the response variable and of random effects
\item likelihood function
\end{itemize}

\newpage
\section{Measures 2} %2.4

\subsection{Cook's Distance} %2.4.1
\begin{itemize}
\item For variance components $\gamma$
\end{itemize}

Diagnostic tool for variance components
\[ C_{\theta i} =(\hat(\theta)_{[i]} - \hat(\theta))^{T}\mbox{cov}( \hat(\theta))^{-1}(\hat(\theta)_{[i]} - \hat(\theta))\]

\subsection{Variance Ratio} %2.4.2
\begin{itemize}
\item For fixed effect parameters $\beta$.
\end{itemize}

\subsection{Cook-Weisberg statistic} %2.4.3
\begin{itemize}
\item For fixed effect parameters $\beta$.
\end{itemize}

\subsection{Andrews-Pregibon statistic} %2.4.4
\begin{itemize}
\item For fixed effect parameters $\beta$.
\end{itemize}
The Andrews-Pregibon statistic $AP_{i}$ is a measure of influence based on the volume of the confidence ellipsoid.
The larger this statistic is for observation $i$, the stronger the influence that observation will have on the model fit.


%---------------------------------------------------------------------------%
\newpage
\section{Haslett's Analysis} %2.5
For fixed effect linear models with correlated error structure Haslett (1999) showed that the effects on
the fixed effects estimate of deleting each observation in turn could be cheaply computed from the fixed effects model predicted residuals.

%-------------------------------------------------------------------------------------------------------------------------------------%
%-------------------------------------------------------------------------------------------------------------------------------------%
%-------------------------------------------------------------------------------------------------Chapter 3------------------------%
%-------------------------------------------------------------------------------------------------------------------------------------%
%-------------------------------------------------------------------------------------------------------------------------------------%

\chapter{Augmented GLMs} 


%---------------------------------------------------------------------------%
% - 3. Augmented GLMS
%---------------------------------------------------------------------------%


Generalized linear models are a generalization of classical linear  models.

\section{Augmented GLMs} %3.1

With the use of h-likihood, a random effected model of the form can be viewed as an `augmented GLM' with the response varaibkes $(y^t, \phi^t_m)^t$, (with $\mu = E(y)$,$ u = E(\phi)$, $var(y) = \theta V (\mu)$.
The augmented linear predictor is \[\eta_{ma}  = (\eta^t, \eta^t_m)^t) = T\omega. \].



%Augmented Generalized linear models.
% Youngjo et al page 154

The subscript $M$ is a label referring to the mean model.
\begin{equation}
\left(%
\begin{array}{c}
  Y \\
  \psi_{M} \\
\end{array}%
\right) = \left(
\begin{array}{cc}
  X & Z \\
  0 & I \\
\end{array}\right) \left(%
\begin{array}{c}
  \beta \\
  \nu \\
\end{array}%
\right)+ e^{*}
\end{equation}


%Augmented Generalized linear models.


The error term $e^{*}$ is normal with mean zero. The variance matrix of the error term is given by
\begin{equation}
\Sigma_{a} = \left(%
\begin{array}{cc}
  \Sigma & 0 \\
  0 & D \\
\end{array}%
\right).
\end{equation}


y_{a} = T \delta + e^{*}
\end{equation}

Weighted least squares equation


% Youngjo et al page 154


\subsection{The Augmented Model Matrix}  %3.2
\begin{equation}
X = \left(%
\begin{array}{cc}
  T & Z \\
  0 & I \\
\end{array}%
\right)
\delta = \left(%
\begin{array}{c}
  \beta  \\
  \nu  \\
\end{array}%
\right)
\end{equation}



\subsection{Importance-Weighted Least-Squares (IWLS)}  %3.3


\subsection{H-Likelihood}




%-------------------------------------------------------------------------------------------------------------------------------------%
%-------------------------------------------------------------------------------------------------------------------------------------%
%-------------------------------------------------------------------------------------------------Chapter 4------------------------%
%-------------------------------------------------------------------------------------------------------------------------------------%
%-------------------------------------------------------------------------------------------------------------------------------------%


%\chapter{Application to Method Comparison Studies} % Chapter 4


%---------------------------------------------------------------------------%
% - 1. Application to MCS
% - 2. Grubbs' Data
% - 3. R implementation
% - 4. Influence measures using R
%---------------------------------------------------------------------------%

\section{Application to MCS} %4.1

Let $\hat{\beta}$ denote the least square estimate of $\beta$
based upon the full set of observations, and let
$\hat{\beta}^{(k)}$ denoted the estimate with the $k^{th}$ case
excluded.


\section{Grubbs' Data} %4.2

For the Grubbs data the $\hat{\beta}$ estimated are
$\hat{\beta}_{0}$ and $\hat{\beta}_{1}$ respectively. Leaving the
fourth case out, i.e. $k=4$ the corresponding estimates are
$\hat{\beta}_{0}^{-4}$ and $\hat{\beta}_{1}^{-4}$


\begin{equation}
Y^{-Q} = \hat{\beta}^{-Q}X^{-Q}
\end{equation}

When considering the regression of case-wise differences and averages, we write $D^{-Q} = \hat{\beta}^{-Q}A^{-Q}$


\newpage

\begin{table}[ht]
\begin{center}
\begin{tabular}{rrrrr}
  \hline
 & F & C & D & A \\
  \hline
1 & 793.80 & 794.60 & -0.80 & 794.20 \\
  2 & 793.10 & 793.90 & -0.80 & 793.50 \\
  3 & 792.40 & 793.20 & -0.80 & 792.80 \\
  4 & 794.00 & 794.00 & 0.00 & 794.00 \\
  5 & 791.40 & 792.20 & -0.80 & 791.80 \\
  6 & 792.40 & 793.10 & -0.70 & 792.75 \\
  7 & 791.70 & 792.40 & -0.70 & 792.05 \\
  8 & 792.30 & 792.80 & -0.50 & 792.55 \\
  9 & 789.60 & 790.20 & -0.60 & 789.90 \\
  10 & 794.40 & 795.00 & -0.60 & 794.70 \\
  11 & 790.90 & 791.60 & -0.70 & 791.25 \\
  12 & 793.50 & 793.80 & -0.30 & 793.65 \\
   \hline
\end{tabular}
\end{center}
\end{table}


\newpage

\begin{equation}
Y^{(k)} = \hat{\beta}^{(k)}X^{(k)}
\end{equation}

Consider two sets of measurements , in this case F and C , with the vectors of case-wise averages $A$ and case-wise differences $D$ respectively. A regression model of differences on averages can be fitted with the view to exploring some characteristics of the data.

When considering the regression of case-wise differences and averages, we write

\begin{equation}
D^{-Q} = \hat{\beta}^{-Q}A^{-Q}
\end{equation}
Let $\hat{\beta}$ denote the least square estimate of $\beta$ based upon the full set of observations, and let $\hat{\beta}^{(k)}$ denoted the estimate with the $k^{th}$ case excluded.

For the Grubbs data the $\hat{\beta}$ estimated are $\hat{\beta}_{0}$ and $\hat{\beta}_{1}$ respectively. Leaving the
fourth case out, i.e. $k=4$ the corresponding estimates are $\hat{\beta}_{0}^{-4}$ and $\hat{\beta}_{1}^{-4}$

\begin{equation}
Y^{(k)} = \hat{\beta}^{(k)}X^{(k)}
\end{equation}

Consider two sets of measurements , in this case F and C , with the vectors of case-wise averages $A$ and case-wise differences $D$ respectively. A regression model of differences on averages can be fitted with the view to exploring some characteristics of the data.

\begin{verbatim}
Call: lm(formula = D ~ A)

Coefficients: (Intercept)            A
  -37.51896      0.04656

\end{verbatim}




When considering the regression of case-wise differences and averages, we write

\begin{equation}
D^{-Q} = \hat{\beta}^{-Q}A^{-Q}
\end{equation}



\subsection{Influence measures using R} %4.4
\texttt{R} provides the following influence measures of each observation.

%Influence measures: This suite of functions can be used to compute
%some of the regression (leave-one-out deletion) diagnostics for
%linear and generalized linear models discussed in Belsley, Kuh and
% Welsch (1980), Cook and Weisberg (1982)



\begin{table}[ht]
\begin{center}
\begin{tabular}{|c|c|c|c|c|c|c|}
  \hline
 & dfb.1\_ & dfb.A & dffit & cov.r & cook.d & hat \\
  \hline
1 & 0.42 & -0.42 & -0.56 & 1.13 & 0.15 & 0.18 \\
  2 & 0.17 & -0.17 & -0.34 & 1.14 & 0.06 & 0.11 \\
  3 & 0.01 & -0.01 & -0.24 & 1.17 & 0.03 & 0.08 \\
  4 & -1.08 & 1.08 & 1.57 & 0.24 & 0.56 & 0.16 \\
  5 & -0.14 & 0.14 & -0.24 & 1.30 & 0.03 & 0.13 \\
  6 & -0.00 & 0.00 & -0.11 & 1.31 & 0.01 & 0.08 \\
  7 & -0.04 & 0.04 & -0.08 & 1.37 & 0.00 & 0.11 \\
  8 & 0.02 & -0.02 & 0.15 & 1.28 & 0.01 & 0.09 \\
  9 & 0.69 & -0.68 & 0.75 & 2.08 & 0.29 & 0.48 \\
  10 & 0.18 & -0.18 & -0.22 & 1.63 & 0.03 & 0.27 \\
  11 & -0.03 & 0.03 & -0.04 & 1.53 & 0.00 & 0.19 \\
  12 & -0.25 & 0.25 & 0.44 & 1.05 & 0.09 & 0.12 \\
   \hline
\end{tabular}
\end{center}
\end{table}



%-------------------------------------------------------------------------------------------------------%
\chapter{Appendices} % Chapter 5
%---------------------------------------------------------------------------------------------------------%
% Appendices
% - The Hat Matrix (5.1)
% - Sherman Morrison Woodbury Formula (5.2)
% -  Hat Matrix applied to MCS (5.3)
% - Cross Validation (Updating standard deviation) (5.4)
% - Updating Estimates (5.5)
% - Lesaffre's paper (5.6)
%---------------------------------------------------------------------------------------------------------%
%------------------------------------------------------------------------%
\newpage
\section{The Hat Matrix} %5.1

The projection matrix $H$ (also known as the hat matrix), is a
well known identity that maps the fitted values $\hat{Y}$ to the
observed values $Y$, i.e. $\hat{Y} = HY$.

\begin{equation}
H =\quad X(X^{T}X)^{-1}X^{T}
\end{equation}

$H$ describes the influence each observed value has on each fitted
value. The diagonal elements of the $H$ are the `leverages', which
describe the influence each observed value has on the fitted value
for that same observation. The residuals ($R$) are related to the
observed values by the following formula:
\begin{equation}
R = (I-H)Y
\end{equation}

The variances of $Y$ and $R$ can be expressed as:
\begin{eqnarray}
\mbox{var}(Y) = H\sigma^{2} \nonumber\\
\mbox{var}(R) = (I-H)\sigma^{2}
\end{eqnarray}

Updating techniques allow an economic approach to recalculating
the projection matrix, $H$, by removing the necessity to refit the
model each time it is updated. However this approach is known for
numerical instability in the case of down-dating.

\section{Sherman Morrison Woodbury Formula} % 5.2

The `Sherman Morrison Woodbury' Formula is a well known result in
linear algebra;
\begin{equation}
(A+a^{T}B)^{-1} \quad = \quad A^{-1}-
A^{-1}a^{T}(I-bA^{-1}a^{T})^{-1}bA^{-1}
\end{equation}

This result is highly useful for analyzing regression diagnostics,
and for matrices inverses in general. Consider a $p \times p$
matrix $X$, from which a row $x_{i}^{T}$ is to be added or
deleted. \citet{CookWeisberg} sets $A = X^{T}X$, $a=-x_{i}^{T}$
and $b=x_{i}^{T}$, and writes the above equation as

\begin{equation}
(X^{T}X \pm x_{i}x_{i}^{T})^{-1} = \quad(X^{T}X )^{-1} \mp \quad
\frac{(X^{T}X)^{-1}(x_{i}x_{i}^{T}(X^{T}X)^{-1}}{1-x_{i}^{T}(X^{T}X)^{-1}x_{i}}
\end{equation}

The projection matrix $H$ (also known as the hat matrix), is a
well known identity that maps the fitted values $\hat{Y}$ to the
observed values $Y$, i.e. $\hat{Y} = HY$.

\begin{equation}
H =\quad X(X^{T}X)^{-1}X^{T}
\end{equation}

$H$ describes the influence each observed value has on each fitted value. The diagonal elements of the $H$ are the `leverages', which describe the influence each observed value has on the fitted value for that same observation. The residuals ($R$) are related to the observed values by the following formula:
\begin{equation}
R = (I-H)Y
\end{equation}

The variances of $Y$ and $R$ can be expressed as:
\begin{eqnarray}
\mbox{var}(Y) = H\sigma^{2} \nonumber\\
\mbox{var}(R) = (I-H)\sigma^{2}
\end{eqnarray}

Updating techniques allow an economic approach to recalculating the projection matrix, $H$, by removing the necessity to refit the model each time it is updated. However this approach is known for
numerical instability in the case of down-dating.



\subsection{Hat Values for MCS regression}

With A as the averages and D as the casewise differences.
\begin{verbatim}
fit = lm(D~A)
\end{verbatim}

\begin{displaymath}
H = A \left(A^\top  A\right)^{-1} A^\top ,
\end{displaymath}

%------------------------------------------------------------------------%
\newpage
\section{Cross Validation} %5.4

Cross validation techniques for linear regression employ the use `leave one out' re-calculations. In such procedures the regression coefficients are estimated for $n-1$ covariates, with the $Q^{th}$ observation omitted.

Let $\hat{\beta}$ denote the least square estimate of $\beta$ based upon the full set of observations, and let
$\hat{\beta}^{-Q}$ denoted the estimate with the $Q^{th}$ case
excluded.


In leave-one-out cross validation, each observation is omitted in turn, and a regression model is fitted on the rest of the data. Cross validation is used to estimate the generalization error of a given model. alternatively it can be used for model selection by determining the candidate model that has the smallest generalization error.


Evidently leave-one-out cross validation has similarities with `jackknifing', a well known statistical technique. However cross validation is used to estimate generalization error, whereas the jackknife technique is used to estimate bias.

\subsection{Cross Validation: Updating standard deviation} %5.4.1

The variance of a data set can be calculated using the following formula.
\begin{equation}
S^{2}=\frac{\sum_{i=1}^{n}(x_{i}^{2})-\frac{(\sum_{i=1}^{n}x_{i})^{2}}{n}}{n-1}
\end{equation}

While using bivariate data, the notation $Sxx$ and $Syy$ shall apply to the variance of $x$ and of $y$ respectively. The covariance term $Sxy$ is given by

\begin{equation}
Sxy=\frac{\sum_{i=1}^{n}(x_{i}y_{i})-\frac{(\sum_{i=1}^{n}x_{i})(\sum_{i=1}^{n}y_{i})}{n}}{n-1}
\end{equation}

Let the observation $j$ be omitted from the data set. The estimates for the variance identities can be updating using minor adjustments to the full sample estimates. Where $(j)$ denotes that the $j$th has been omitted, these identities are

\begin{equation}
Sxx^{(j)}=\frac{\sum_{i=1}^{n}(x_{i}^{2})-(x_{j})^{2}-\frac{((\sum_{i=1}^{n}x_{i})-x_{j})^{2}}{n-1}}{n-2}
\end{equation}
\begin{equation}
Syy^{(j)}=\frac{\sum_{i=1}^{n}(y_{i}^{2})-(y_{j})^{2}-\frac{((\sum_{i=1}^{n}y_{i})-y_{j})^{2}}{n-1}}{n-2}
\end{equation}
\begin{equation}
Sxy^{(j)}=\frac{\sum_{i=1}^{n}(x_{i}y_{i})-(y_{j}x_{j})-\frac{((\sum_{i=1}^{n}x_{i})-x_{j})(\sum_{i=1}^{n}y_{i})-y_{k})}{n-1}}{n-2}
\end{equation}

The updated estimate for the slope is therefore
\begin{equation}
\hat{\beta}_{1}^{(j)}=\frac{Sxy^{(j)}}{Sxx^{(j)}}
\end{equation}

It is necessary to determine the mean for $x$ and $y$ of the
remaining $n-1$ terms
\begin{equation}
\bar{x}^{(j)}=\frac{(\sum_{i=1}^{n}x_{i})-(x_{j})}{n-1},
\end{equation}

\begin{equation}
\bar{y}^{(j)}=\frac{(\sum_{i=1}^{n}y_{i})-(y_{j})}{n-1}.
\end{equation}

The updated intercept estimate is therefore

\begin{equation}
\hat{\beta}_{0}^{(j)}=\bar{y}^{(j)}-\hat{\beta}_{1}^{(j)}\bar{x}^{(j)}.
\end{equation}

%------------------------------------------------------------------------%
\newpage
\section{Updating Estimates} %5.5

\subsection{Updating of Regression Estimates}
Updating techniques are used in regression analysis to add or delete rows from a model, allowing the analyst the effect of the observation associated with that row. In time series problems, there will be scientific interest in the changing relationship between variables. In cases where there a single row is to be added or deleted, the procedure used is equivalent to a geometric rotation of a plane.

Updating techniques are used in regression analysis to add or delete rows from a model, allowing the analyst the effect of the observation associated with that row.

\subsection{Updating Standard deviation}
A simple, but useful, example of updating is the updating of the standard deviation when an observation is omitted, as practised in statistical process control analyzes. From first principles, the variance of a data set can be calculated using the following formula.
\begin{equation}
S^{2}=\frac{\sum_{i=1}^{n}(x_{i}^{2})-\frac{(\sum_{i=1}^{n}x_{i})^{2}}{n}}{n-1}
\end{equation}

While using bivariate data, the notation $Sxx$ and $Syy$ shall apply hither to the variance of $x$ and of $y$ respectively. The covariance term $Sxy$ is given by

\begin{equation}
Sxy=\frac{\sum_{i=1}^{n}(x_{i}y_{i})-\frac{(\sum_{i=1}^{n}x_{i})(\sum_{i=1}^{n}y_{i})}{n}}{n-1}.
\end{equation}

\subsection{Updating of Regression Estimates}
Updating techniques are used in regression analysis to add or
delete rows from a model, allowing the analyst the effect of the
observation associated with that row. In time series problems,
there will be scientific interest in the changing relationship
between variables. In cases where there a single row is to be
added or deleted, the procedure used is equivalent to a geometric
rotation of a plane.

Consider a $p \times p$ matrix $X$, from which a row $x_{i}^{T}$
is to be added or deleted. \citet{CookWeisberg} sets $A = X^{T}X$,
$a=-x_{i}^{T}$ and $b=x_{i}^{T}$, and writes the above equation as

\begin{equation}
(X^{T}X \pm x_{i}x_{i}^{T})^{-1} = \quad(X^{T}X )^{-1} \mp \quad
\frac{(X^{T}X)^{-1}(x_{i}x_{i}^{T}(X^{T}X)^{-1}}{1-x_{i}^{T}(X^{T}X)^{-1}x_{i}}
\end{equation}

\subsection{Updating Regression Estimates}
Let the observation $j$ be omitted from the data set. The estimates for the variance identities can be updating using minor adjustments to the full sample estimates. Where $(j)$ denotes that the $j$th has been omitted, these identities are

\begin{equation}
Sxx^{(j)}=\frac{\sum_{i=1}^{n}(x_{i}^{2})-(x_{j})^{2}-\frac{((\sum_{i=1}^{n}x_{i})-x_{j})^{2}}{n-1}}{n-2}
\end{equation}
\begin{equation}
Syy^{(j)}=\frac{\sum_{i=1}^{n}(y_{i}^{2})-(y_{j})^{2}-\frac{((\sum_{i=1}^{n}y_{i})-y_{j})^{2}}{n-1}}{n-2}
\end{equation}
\begin{equation}
Sxy^{(j)}=\frac{\sum_{i=1}^{n}(x_{i}y_{i})-(y_{j}x_{j})-\frac{((\sum_{i=1}^{n}x_{i})-x_{j})(\sum_{i=1}^{n}y_{i})-y_{k})}{n-1}}{n-2}
\end{equation}

The updated estimate for the slope is therefore
\begin{equation}
\hat{\beta}_{1}^{(j)}=\frac{Sxy^{(j)}}{Sxx^{(j)}}
\end{equation}

It is necessary to determine the mean for $x$ and $y$ of the
remaining $n-1$ terms
\begin{equation}
\bar{x}^{(j)}=\frac{(\sum_{i=1}^{n}x_{i})-(x_{j})}{n-1},
\end{equation}

\begin{equation}
\bar{y}^{(j)}=\frac{(\sum_{i=1}^{n}y_{i})-(y_{j})}{n-1}.
\end{equation}

The updated intercept estimate is therefore

\begin{equation}
\hat{\beta}_{0}^{(j)}=\bar{y}^{(j)}-\hat{\beta}_{1}^{(j)}\bar{x}^{(j)}.
\end{equation}

\subsection{Inference on intercept and slope}
\begin{equation}
\hat{\beta_{1}} \pm t_{(\alpha, n-2) }
\sqrt{\frac{S^2}{(n-1)S^{2}_{x}}}
\end{equation}

\begin{equation}
\frac{\hat{\beta_{0}}-\beta_{0}}{SE(\hat{\beta_{0}})}
\end{equation}
\begin{equation}
\frac{\hat{\beta_{1}}-\beta_{1}}{SE(\hat{\beta_{0}})}
\end{equation}


\subsubsection{Inference on correlation coefficient} This test of
the slope is coincidentally the equivalent of a test of the
correlation of the $n$ observations of $X$ and $Y$.
\begin{eqnarray}
H_{0}: \rho_{XY} = 0 \nonumber \\
H_{A}: \rho_{XY} \ne 0 \nonumber \\
\end{eqnarray}


\printindex
\bibliographystyle{chicago}
\bibliography{DB-txfrbib}

\end{document} 
