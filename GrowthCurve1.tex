%- https://en.wikipedia.org/wiki/Growth_curve_(statistics)

\section*{Growth Curve}
The growth curve model in statistics is a specific multivariate linear model, also known as GMANOVA (Generalized Multivariate ANalysis-Of-VAriance).[1] It generalizes MANOVA by allowing post-matrices, as seen in the definition.


\section*{Definition}
Growth curve model:[2] Let X be a p×n matrix, A a p×q matrix with q ≤ p, B a q×k matrix,C a k×n matrix with rank(C) + p ≤ n and let Σ be a positive-definite p×p matrix. Then

\[{\displaystyle X=ABC+\Sigma ^{1/2}E} X=ABC+\Sigma^{1/2}E\]
defines the growth curve model, where A and C are known, B and Σ are unknown, and E is a random matrix distributed as Np,n(0,Ip,n).

This differs from standard MANOVA by the addition of C, a "postmatrix".[3]



The growth curve model was invented by Potthoff and Roy in 1964;[3] they used it to analyze repeated measurements of animals or humans to obtain a biological growth curve.
GMANOVA is frequently used for the analysis of surveys, clinical trials, and agricultural data,[4] as well as more recently in the context of Radar adaptive detection.[5][6]
