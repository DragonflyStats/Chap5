2012 June 14th ( Barnhart et Al)
Replicate Measurements (Barnhart Pg 6)

Repeatability (Barnhart pg 7)
ISOs definition: Closeness of agreement netween measures under the same conditions. (i.e. True Replicates)
 

2012 May 24
The matter of how well two methods of measurement are said to be “in agreement” is a frequently posed question in statistical literature. A useful, and broadly consistent, set of definitions of what this “agreement” entail is put forth by Barnhart et al and Roy (2009). 
As pointed out by earlier contributors to the subject ( commonly referred to as “Method Comparison Studies”)
Shared with previous contributions (Bland and Altman, Carstensen) is the condition that there should no systematic  tendency for one of the methods to consistently provide a value higher that than of the other method. If such a tendency did exist, we would refer to it as an inter-method bias.
In earlier literature, the emphasis was placed up on single measurements simultaneously by each of the methods of measurement. Several different approaches, such as the Bland-Altman plot, and Orthogonal Regression (a special case of Deming Regression where the residual variances are assumed to be equal) have been proposed. Arguably, for the single replicate case, the established methodologies are sufficient for assessing agreement between two methods.
In subsequent contributions, the matter of assessing agreement in the presence  of replicate measurements was addressed. Some approaches extended already established approaches (Bland-Altam 1999).  Other contributions were based on methodologies not seen previously in Method comparison Study Literature  (for example, Carstensen et al 2008 and Roy 2009, using LME models). 
A review of recent literature demonstrates how useful and effective the use of LME models are.


MAY 2012 : Research Notes
Roy (2009) proposes a suite of hypothesis tests for assessing the agreement of two methods of measurement, when replicate measurements are obtained for each item, using a LME approach. (An item would commonly be a patient).  Two methods of measurement can be said to be in agreement if there is no significant difference between in three key respects. Firstly, there is no inter-method bias between the two methods, i.e. there is no persistent tendency for one method to give higher values than the other.
Secondly, both methods of measurement have the same  within-subject variability. In such a case the variance of the replicate measurements would consistent for both methods.
Lastly, the methods have equal between-subject variability.  Put simply, for the mean measurements for each case, the variances of the mean measurements from both methods are equal.
Testing for Inter-method Bias
Firstly, a practitioner would investigate whether a significant inter-method bias is present between the methods. This bias is specified as a fixed effect in the LME model.  For a practitioner who has a reasonable level of competency in R and undergraduate statistics (in particular simple linear regression model) this is a straight-forward procedure.
Reference Model (Ref.Fit)
Conventionally LME models can be tested using Likelihood Ratio Tests, wherein a reference model is compared to a nested model.
> Ref.Fit = lme(y ~ meth-1, data = dat,   #Symm , Symm#
+     random = list(item=pdSymm(~ meth-1)), 
+     weights=varIdent(form=~1|meth),
+     correlation = corSymm(form=~1 | item/repl), 
+     method="ML")

Roy(2009) presents two nested models that specify the condition of equality as required, with a third nested model for an additional test. There three formulations share the same structure, and can be specified by making slight alterations of the code for the Reference Model.
Nested Model (Between-Item Variability)
> NMB.fit  = lme(y ~ meth-1, data = dat,   #CS , Symm#
+     random = list(item=pdCompSymm(~ meth-1)),
+     correlation = corSymm(form=~1 | item/repl), 
+     method="ML")


Nested Model (Within –item Variability)
> NMW.fit = lme(y ~ meth-1, data = dat,   #Symm , CS# 
+     random = list(item=pdSymm(~ meth-1)),
+     weights=varIdent(form=~1|meth), 
+     correlation = corCompSymm(form=~1 | item/repl), 
+     method="ML")




Nested Model (Overall Variability)
Additionally there is a third nested model, that can be used to test overall variability, substantively a a joint test for between-item and within-item variability. The motivation for including such a test in the suite is not clear, although it does circumvent the need for multiple comparison procedures in certain circumstances, hence providing a simplified procedure for non-statisticians.
   > NMO.fit = lme(y ~ meth-1, data = dat,   #CS , CS# 
+     random = list(item=pdCompSymm(~ meth-1)), 
+     correlation = corCompSymm(form=~1 | item/repl), 
+     method="ML")


ANOVAs  for  Original Fits
The likelihood Ratio test is very simple to implement in R. All that is required it to specify the reference model and the relevant nested mode as arguments to the command anova().
The figure below displays the three tests described by Roy (2009).
> testB    = anova(Ref.Fit,NMB.fit)                        # Between-Subject Variabilities
> testW   = anova(Ref.Fit,NMW.fit)                        # Within-Subject Variabilities
> testO     = anova(Ref.Fit,NMO.fit)                        # Overall Variabilities

  
 


 


Using REML Fitting

Noticeably Roy (2009) uses ML estimation when specifying the LME models. No explanation is given, although plausibly it is due to the constraints of the computational environment being used.
Both West et al (2010) and Pinheiro and Bates (2000) compare ML and REML estimation, describing what types of tests are appropriate for each.  When variance components are being tested, REML estimation is in fact the correct approach.
> fit1r = lme(y ~ meth-1, data = dat,   #Symm , Symm#
+     random = list(item=pdSymm(~ meth-1)), 
+     weights=varIdent(form=~1|meth),
+     correlation = corSymm(form=~1 | item/repl), 
+     method="REML")

     
 


> fit2r = lme(y ~ meth-1, data = dat,   #CS , Symm#
+     random = list(item=pdCompSymm(~ meth-1)),
+     correlation = corSymm(form=~1 | item/repl), 
+     method="REML")



  
> fit3r = lme(y ~ meth-1, data = dat,   #Symm , CS# 
+     random = list(item=pdSymm(~ meth-1)),
+     weights=varIdent(form=~1|meth), 
+     correlation = corCompSymm(form=~1 | item/repl), 
+     method="REML")



> fit4r = lme(y ~ meth-1, data = dat,   #CS , CS# 
+     random = list(item=pdCompSymm(~ meth-1)), 
+     correlation = corCompSymm(form=~1 | item/repl), 
+     method="REML")



> test1r = anova(fit1r,fit2r)                     # Between-Subject Variabilities
> test2r = anova(fit1r,fit3r)                    # Within-Subject Variabilities
> test3r = anova(fit1r,fit4r)                    # Overall Variabilities   



> fit1bias = lme(y ~ meth, data = dat,   #Symm , Symm#
+     random = list(item=pdSymm(~ meth-1)), 
+     weights=varIdent(form=~1|meth),
+     correlation = corSymm(form=~1 | item/repl), 
+     method="ML")



 
 

Comparison of ML and REML fits
Fit 1 (ML)

Dataset: Blood RS

Fixed : 127.3126 , 143.0275

AIC: 4075.594

Between Subject Variability

 	Fit1r (REML)

Dataset: Blood RS

Fixed : 127.3126 , 143.0275

AIC: 4068.172

Between Subject Variability

 

 


# Systolic blood pressure measurements made 
# simultaneously by two observers (J and R) 
# and an automatic blood pressure measuring
# machine (S), each making three observations 
# in quick succession (supplied by Dr E O'Brien)

Blood = matrix(data=c(100, 106, 107, 98, 98, 111, 122, 128, 124, 108, 110, 108, 108, 112, 110, 121, 
127, 128, 76, 84, 82, 76, 88, 82, 95, 94, 98,108, 104, 104, 110, 100, 106, 127, 127, 135,124, 112, 112, 128, 112, 114, 140, 131, 124,122, 140, 124, 124, 140, 126, 139, 142, 136,116, 108, 102, 118, 110, 102, 122, 112, 112,114, 110, 112, 112, 108, 112, 130, 129, 135,100, 108, 112, 100, 106, 112, 119, 122, 122,108, 92, 100, 108, 98, 100, 126, 113, 111,100, 106, 104, 102, 108, 106, 107, 113, 111,108, 112, 122, 108, 116, 120, 123, 125, 125,112, 112, 110, 114, 112, 110, 131, 129, 122,104, 108, 104, 104, 108, 104, 123, 126, 114,106, 108, 102, 104, 106, 102, 127, 119, 126,122, 122, 114, 118, 122, 114, 142, 133, 137,100, 102, 102, 102, 102, 100, 104, 116, 115,118, 118, 120, 116, 118, 118, 117, 113, 112,140, 134, 138, 138, 136, 134, 139, 127, 113,150, 148, 144, 148, 146, 144, 143, 155, 133,166, 154, 154, 164, 154, 148, 181, 170, 166,148, 156, 134, 136, 154, 132, 149, 156, 140,
174, 172, 166, 170, 170, 164, 173, 170, 154,174, 166, 150, 174, 166, 154, 160, 155, 170,140, 144, 144, 140, 144, 144, 158, 152, 154,128, 134, 130, 128, 134, 130, 139, 144, 141,
146, 138, 140, 146, 138, 138, 153, 150, 154,146, 152, 148, 146, 152, 148, 138, 144, 131,220, 218, 220, 220, 218, 220, 228, 228, 226,208, 200, 192, 204, 200, 190, 190, 183, 184,
94, 84, 86, 94, 84, 88, 103, 99, 106,114, 124, 116, 112, 126, 118, 131, 131, 124,126, 120, 122, 124, 120, 120, 131, 123, 124,124, 124, 132, 126, 126, 120, 126, 129, 125,110, 120, 128, 110, 122, 126, 121, 114, 125,90, 90, 94, 88, 88, 94, 97, 94, 96,106, 106, 110, 106, 108, 110, 116, 121, 127,218, 202, 208, 218, 200, 206, 215, 201, 207,130, 128, 130, 128, 126, 128, 141, 133, 146,136, 136, 130, 136, 138, 128, 153, 143, 138,100, 96, 88, 100, 96, 86, 113, 107, 102,100, 98, 88, 100, 98, 88, 109, 105, 97,
124, 116, 122, 126, 116, 122, 145, 102, 137,164, 168, 154, 164, 168, 154, 192, 178, 171,100, 102, 100, 100, 104, 102, 112, 116, 116,136, 126, 122, 136, 124, 122, 152, 144, 147,114, 108, 122, 114, 108, 122, 141, 141, 137,148, 120, 132, 146, 130, 132, 206, 188, 166,160, 150, 148, 160, 152, 146, 151, 147, 136,84, 92, 98, 86, 92, 98, 112, 125, 124,156, 162, 152, 156, 158, 152, 162, 165, 189,110, 98, 98, 108, 100, 98, 117, 118, 109,100, 106, 106, 100, 108, 108, 119, 131, 124,100, 102, 94, 100, 102, 96, 136, 116, 113,86, 74, 76, 88, 76, 76, 112, 115, 104,106, 100, 110, 106, 100, 108, 120, 118, 132,108, 110, 106, 106, 118, 106, 117, 118, 115,168, 188, 178, 170, 188, 182, 194, 191, 196,166, 150, 154, 164, 150, 154, 167, 160, 161,146, 142, 132, 144, 142, 130, 173, 161, 154,204, 198, 188, 206, 198, 188, 228, 218, 189,96, 94, 86, 96, 94, 84, 77, 89, 101,134, 126, 124, 132, 126, 124, 154, 156, 141,138, 144, 140, 140, 142, 138, 154, 155, 148,134, 136, 142, 136, 134, 140, 145, 154, 166,156, 160, 154, 156, 162, 156, 200, 180, 179,124, 138, 138, 122, 140, 136, 188, 147, 139,114, 110, 114, 112, 114, 114, 149, 217, 192,112, 116, 122, 112, 114, 124, 136, 132, 133,112, 116, 134, 114, 114, 136, 128, 125, 142,202, 220, 228, 200, 220, 226, 204, 222, 224,132, 136, 134, 134, 136, 132, 184, 187, 192,158, 162, 152, 158, 164, 150, 163, 160, 152,88, 76, 88, 90, 76, 86, 93, 88, 88,170, 174, 176, 172, 174, 178, 178, 181, 181,182, 176, 180, 184, 174, 178, 202, 199, 195,112, 114, 124, 112, 112, 126, 162, 166, 148,120, 118, 120, 118, 116, 120, 227, 227, 219,110, 108, 106, 110, 108, 106, 133, 127, 126,112, 112, 106, 112, 110, 106, 202, 190, 213,154, 134, 130, 156, 136, 132, 158, 121, 134,
116, 112, 94, 118, 114, 96, 124, 149, 137,108, 110, 114, 106, 110, 114, 114, 118, 126,106, 98, 100, 104, 100, 100, 137, 135, 134,122, 112, 112, 122, 114, 114, 121, 123, 128), 
nrow = 85, ncol = 9, byrow = TRUE,
dimnames = list(NULL, c("J1","J2","J3","R1","R2","R3","S1","S2","S3")) )



#####################################################################
#Preparing the Blood Data
library(nlme)
blood = groupedData( y ~ meth | item ,
    data = data.frame( y = c(Blood), item = c(row(Blood)),
        meth = rep(c("J","R","S"), rep(nrow(Blood)*3, 3)),
        repl = rep(rep(c(1:3), rep(nrow(Blood), 3)), 3) ),
    labels = list(BP = "Systolic Blood Pressure", method = "Measurement Device"),
    order.groups = FALSE )
 
#pick out two of the three methods ( use J and S ) 
   
dat = subset(blood, subset = meth != "R")

#####################################################################
#Roy's Candidate Models

    
fit1 = lme(y ~ meth-1, data = dat,   #Symm , Symm#
    random = list(item=pdSymm(~ meth-1)), 
    weights=varIdent(form=~1|meth),
    correlation = corSymm(form=~1 | item/repl), 
    method="ML")
    
fit2 = lme(y ~ meth-1, data = dat,   #CS , Symm#
    random = list(item=pdCompSymm(~ meth-1)),
    correlation = corSymm(form=~1 | item/repl), 
    method="ML")
    
fit3 = lme(y ~ meth-1, data = dat,   #Symm , CS# 
    random = list(item=pdSymm(~ meth-1)),
    weights=varIdent(form=~1|meth), 
    correlation = corCompSymm(form=~1 | item/repl), 
    method="ML")
    
fit4 = lme(y ~ meth-1, data = dat,   #CS , CS# 
    random = list(item=pdCompSymm(~ meth-1)), 
    correlation = corCompSymm(form=~1 | item/repl), 
    method="ML")
    
#####################################################################
#ANOVAs
test1 = anova(fit1,fit2) # Between-Subject Variabilities
test2 = anova(fit1,fit3) # Within-Subject Variabilities
test3 = anova(fit1,fit4) # Overall Variabilities


#####################################################################
#Using REML Fitting


fit1r = lme(y ~ meth-1, data = dat,   #Symm , Symm#
    random = list(item=pdSymm(~ meth-1)), 
    weights=varIdent(form=~1|meth),
    correlation = corSymm(form=~1 | item/repl), 
    method="REML")
    
 


MAY 2012 : Research Notes
Roy (2009) proposes a suite of hypothesis tests for assessing the agreement of two methods of measurement, when replicate measurements are obtained for each item, using a LME approach. (An item would commonly be a patient).  Two methods of measurement can be said to be in agreement if there is no significant difference between in three key respects. Firstly, there is no inter-method bias between the two methods, i.e. there is no persistent tendency for one method to give higher values than the other.
Secondly, both methods of measurement have the same  within-subject variability. In such a case the variance of the replicate measurements would consistent for both methods.
Lastly, the methods have equal between-subject variability.  Put simply, for the mean measurements for each case, the variances of the mean measurements from both methods are equal.
Testing for Inter-method Bias
Firstly, a practitioner would investigate whether a significant inter-method bias is present between the methods. This bias is specified as a fixed effect in the LME model.  For a practitioner who has a reasonable level of competency in R and undergraduate statistics (in particular simple linear regression model) this is a straight-forward procedure.
Reference Model (Ref.Fit)
Conventionally LME models can be tested using Likelihood Ratio Tests, wherein a reference model is compared to a nested model.
> Ref.Fit = lme(y ~ meth-1, data = dat,   #Symm , Symm#
+     random = list(item=pdSymm(~ meth-1)), 
+     weights=varIdent(form=~1|meth),
+     correlation = corSymm(form=~1 | item/repl), 
+     method="ML")

Roy(2009) presents two nested models that specify the condition of equality as required, with a third nested model for an additional test. There three formulations share the same structure, and can be specified by making slight alterations of the code for the Reference Model.
Nested Model (Between-Item Variability)
> NMB.fit  = lme(y ~ meth-1, data = dat,   #CS , Symm#
+     random = list(item=pdCompSymm(~ meth-1)),
+     correlation = corSymm(form=~1 | item/repl), 
+     method="ML")


Nested Model (Within –item Variability)
> NMW.fit = lme(y ~ meth-1, data = dat,   #Symm , CS# 
+     random = list(item=pdSymm(~ meth-1)),
+     weights=varIdent(form=~1|meth), 
+     correlation = corCompSymm(form=~1 | item/repl), 
+     method="ML")




Nested Model (Overall Variability)
Additionally there is a third nested model, that can be used to test overall variability, substantively a a joint test for between-item and within-item variability. The motivation for including such a test in the suite is not clear, although it does circumvent the need for multiple comparison procedures in certain circumstances, hence providing a simplified procedure for non-statisticians.
   > NMO.fit = lme(y ~ meth-1, data = dat,   #CS , CS# 
+     random = list(item=pdCompSymm(~ meth-1)), 
+     correlation = corCompSymm(form=~1 | item/repl), 
+     method="ML")


ANOVAs  for  Original Fits
The likelihood Ratio test is very simple to implement in R. All that is required it to specify the reference model and the relevant nested mode as arguments to the command anova().
The figure below displays the three tests described by Roy (2009).
> testB    = anova(Ref.Fit,NMB.fit)                        # Between-Subject Variabilities
> testW   = anova(Ref.Fit,NMW.fit)                        # Within-Subject Variabilities
> testO     = anova(Ref.Fit,NMO.fit)                        # Overall Variabilities

  
 


 


Using REML Fitting

Noticeably Roy (2009) uses ML estimation when specifying the LME models. No explanation is given, although plausibly it is due to the constraints of the computational environment being used.
Both West et al (2010) and Pinheiro and Bates (2000) compare ML and REML estimation, describing what types of tests are appropriate for each.  When variance components are being tested, REML estimation is in fact the correct approach.
> fit1r = lme(y ~ meth-1, data = dat,   #Symm , Symm#
+     random = list(item=pdSymm(~ meth-1)), 
+     weights=varIdent(form=~1|meth),
+     correlation = corSymm(form=~1 | item/repl), 
+     method="REML")

     
 


> fit2r = lme(y ~ meth-1, data = dat,   #CS , Symm#
+     random = list(item=pdCompSymm(~ meth-1)),
+     correlation = corSymm(form=~1 | item/repl), 
+     method="REML")



  
> fit3r = lme(y ~ meth-1, data = dat,   #Symm , CS# 
+     random = list(item=pdSymm(~ meth-1)),
+     weights=varIdent(form=~1|meth), 
+     correlation = corCompSymm(form=~1 | item/repl), 
+     method="REML")



> fit4r = lme(y ~ meth-1, data = dat,   #CS , CS# 
+     random = list(item=pdCompSymm(~ meth-1)), 
+     correlation = corCompSymm(form=~1 | item/repl), 
+     method="REML")



> test1r = anova(fit1r,fit2r)                     # Between-Subject Variabilities
> test2r = anova(fit1r,fit3r)                    # Within-Subject Variabilities
> test3r = anova(fit1r,fit4r)                    # Overall Variabilities   



> fit1bias = lme(y ~ meth, data = dat,   #Symm , Symm#
+     random = list(item=pdSymm(~ meth-1)), 
+     weights=varIdent(form=~1|meth),
+     correlation = corSymm(form=~1 | item/repl), 
+     method="ML")



 
 

Comparison of ML and REML fits
Fit 1 (ML)

Dataset: Blood RS

Fixed : 127.3126 , 143.0275

AIC: 4075.594

Between Subject Variability

 	Fit1r (REML)

Dataset: Blood RS

Fixed : 127.3126 , 143.0275

AIC: 4068.172

Between Subject Variability

 

 


# Systolic blood pressure measurements made 
# simultaneously by two observers (J and R) 
# and an automatic blood pressure measuring
# machine (S), each making three observations 
# in quick succession (supplied by Dr E O'Brien)

Blood = matrix(data=c(100, 106, 107, 98, 98, 111, 122, 128, 124, 108, 110, 108, 108, 112, 110, 121, 
127, 128, 76, 84, 82, 76, 88, 82, 95, 94, 98,108, 104, 104, 110, 100, 106, 127, 127, 135,124, 112, 112, 128, 112, 114, 140, 131, 124,122, 140, 124, 124, 140, 126, 139, 142, 136,116, 108, 102, 118, 110, 102, 122, 112, 112,114, 110, 112, 112, 108, 112, 130, 129, 135,100, 108, 112, 100, 106, 112, 119, 122, 122,108, 92, 100, 108, 98, 100, 126, 113, 111,100, 106, 104, 102, 108, 106, 107, 113, 111,108, 112, 122, 108, 116, 120, 123, 125, 125,112, 112, 110, 114, 112, 110, 131, 129, 122,104, 108, 104, 104, 108, 104, 123, 126, 114,106, 108, 102, 104, 106, 102, 127, 119, 126,122, 122, 114, 118, 122, 114, 142, 133, 137,100, 102, 102, 102, 102, 100, 104, 116, 115,118, 118, 120, 116, 118, 118, 117, 113, 112,140, 134, 138, 138, 136, 134, 139, 127, 113,150, 148, 144, 148, 146, 144, 143, 155, 133,166, 154, 154, 164, 154, 148, 181, 170, 166,148, 156, 134, 136, 154, 132, 149, 156, 140,
174, 172, 166, 170, 170, 164, 173, 170, 154,174, 166, 150, 174, 166, 154, 160, 155, 170,140, 144, 144, 140, 144, 144, 158, 152, 154,128, 134, 130, 128, 134, 130, 139, 144, 141,
146, 138, 140, 146, 138, 138, 153, 150, 154,146, 152, 148, 146, 152, 148, 138, 144, 131,220, 218, 220, 220, 218, 220, 228, 228, 226,208, 200, 192, 204, 200, 190, 190, 183, 184,
94, 84, 86, 94, 84, 88, 103, 99, 106,114, 124, 116, 112, 126, 118, 131, 131, 124,126, 120, 122, 124, 120, 120, 131, 123, 124,124, 124, 132, 126, 126, 120, 126, 129, 125,110, 120, 128, 110, 122, 126, 121, 114, 125,90, 90, 94, 88, 88, 94, 97, 94, 96,106, 106, 110, 106, 108, 110, 116, 121, 127,218, 202, 208, 218, 200, 206, 215, 201, 207,130, 128, 130, 128, 126, 128, 141, 133, 146,136, 136, 130, 136, 138, 128, 153, 143, 138,100, 96, 88, 100, 96, 86, 113, 107, 102,100, 98, 88, 100, 98, 88, 109, 105, 97,
124, 116, 122, 126, 116, 122, 145, 102, 137,164, 168, 154, 164, 168, 154, 192, 178, 171,100, 102, 100, 100, 104, 102, 112, 116, 116,136, 126, 122, 136, 124, 122, 152, 144, 147,114, 108, 122, 114, 108, 122, 141, 141, 137,148, 120, 132, 146, 130, 132, 206, 188, 166,160, 150, 148, 160, 152, 146, 151, 147, 136,84, 92, 98, 86, 92, 98, 112, 125, 124,156, 162, 152, 156, 158, 152, 162, 165, 189,110, 98, 98, 108, 100, 98, 117, 118, 109,100, 106, 106, 100, 108, 108, 119, 131, 124,100, 102, 94, 100, 102, 96, 136, 116, 113,86, 74, 76, 88, 76, 76, 112, 115, 104,106, 100, 110, 106, 100, 108, 120, 118, 132,108, 110, 106, 106, 118, 106, 117, 118, 115,168, 188, 178, 170, 188, 182, 194, 191, 196,166, 150, 154, 164, 150, 154, 167, 160, 161,146, 142, 132, 144, 142, 130, 173, 161, 154,204, 198, 188, 206, 198, 188, 228, 218, 189,96, 94, 86, 96, 94, 84, 77, 89, 101,134, 126, 124, 132, 126, 124, 154, 156, 141,138, 144, 140, 140, 142, 138, 154, 155, 148,134, 136, 142, 136, 134, 140, 145, 154, 166,156, 160, 154, 156, 162, 156, 200, 180, 179,124, 138, 138, 122, 140, 136, 188, 147, 139,114, 110, 114, 112, 114, 114, 149, 217, 192,112, 116, 122, 112, 114, 124, 136, 132, 133,112, 116, 134, 114, 114, 136, 128, 125, 142,202, 220, 228, 200, 220, 226, 204, 222, 224,132, 136, 134, 134, 136, 132, 184, 187, 192,158, 162, 152, 158, 164, 150, 163, 160, 152,88, 76, 88, 90, 76, 86, 93, 88, 88,170, 174, 176, 172, 174, 178, 178, 181, 181,182, 176, 180, 184, 174, 178, 202, 199, 195,112, 114, 124, 112, 112, 126, 162, 166, 148,120, 118, 120, 118, 116, 120, 227, 227, 219,110, 108, 106, 110, 108, 106, 133, 127, 126,112, 112, 106, 112, 110, 106, 202, 190, 213,154, 134, 130, 156, 136, 132, 158, 121, 134,
116, 112, 94, 118, 114, 96, 124, 149, 137,108, 110, 114, 106, 110, 114, 114, 118, 126,106, 98, 100, 104, 100, 100, 137, 135, 134,122, 112, 112, 122, 114, 114, 121, 123, 128), 
nrow = 85, ncol = 9, byrow = TRUE,
dimnames = list(NULL, c("J1","J2","J3","R1","R2","R3","S1","S2","S3")) )



#####################################################################
#Preparing the Blood Data
library(nlme)
blood = groupedData( y ~ meth | item ,
    data = data.frame( y = c(Blood), item = c(row(Blood)),
        meth = rep(c("J","R","S"), rep(nrow(Blood)*3, 3)),
        repl = rep(rep(c(1:3), rep(nrow(Blood), 3)), 3) ),
    labels = list(BP = "Systolic Blood Pressure", method = "Measurement Device"),
    order.groups = FALSE )
 
#pick out two of the three methods ( use J and S ) 
   
dat = subset(blood, subset = meth != "R")

#####################################################################
#Roy's Candidate Models

    
fit1 = lme(y ~ meth-1, data = dat,   #Symm , Symm#
    random = list(item=pdSymm(~ meth-1)), 
    weights=varIdent(form=~1|meth),
    correlation = corSymm(form=~1 | item/repl), 
    method="ML")
    
fit2 = lme(y ~ meth-1, data = dat,   #CS , Symm#
    random = list(item=pdCompSymm(~ meth-1)),
    correlation = corSymm(form=~1 | item/repl), 
    method="ML")
    
fit3 = lme(y ~ meth-1, data = dat,   #Symm , CS# 
    random = list(item=pdSymm(~ meth-1)),
    weights=varIdent(form=~1|meth), 
    correlation = corCompSymm(form=~1 | item/repl), 
    method="ML")
    
fit4 = lme(y ~ meth-1, data = dat,   #CS , CS# 
    random = list(item=pdCompSymm(~ meth-1)), 
    correlation = corCompSymm(form=~1 | item/repl), 
    method="ML")
    
#####################################################################
#ANOVAs
test1 = anova(fit1,fit2) # Between-Subject Variabilities
test2 = anova(fit1,fit3) # Within-Subject Variabilities
test3 = anova(fit1,fit4) # Overall Variabilities


#####################################################################
#Using REML Fitting


fit1r = lme(y ~ meth-1, data = dat,   #Symm , Symm#
    random = list(item=pdSymm(~ meth-1)), 
    weights=varIdent(form=~1|meth),
    correlation = corSymm(form=~1 | item/repl), 
    method="REML")
    
 

