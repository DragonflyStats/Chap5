\documentclass[Chap5amain.tex]{subfiles}
\begin{document}
	%---------------------------------------------------------------------------%
	
\section{Measures of Influence} %1.16

The impact of an observation on a regression fitting can be determined by the difference between the estimated regression coefficient of a model with all observations and the estimated coefficient when the particular observation is deleted. The measure DFBETA is the studentized value of this difference.

Influence arises at two stages of the LME model. Firstly when $V$ is estimated by $\hat{V}$, and subsequent
estimations of the fixed and random regression coefficients $\beta$ and $u$, given $\hat{V}$.


\subsection{DFFITS} %1.16.1
DFFITS is a statistical measured designed to a show how influential an observation is in a statistical model. It is closely related to the studentized residual.
\begin{displaymath} DFFITS = {\widehat{y_i} -
\widehat{y_{i(k)}} \over s_{(k)} \sqrt{h_{ii}}} \end{displaymath}


\subsection{PRESS} %1.16.2
The prediction residual sum of squares (PRESS) is an value associated with this calculation. When fitting linear models, PRESS can be used as a criterion for model selection, with smaller values indicating better model fits.
\begin{equation}
PRESS = \sum(y-y^{(k)})^2
\end{equation}


\begin{itemize}
\item $e_{-Q} = y_{Q} - x_{Q}\hat{\beta}^{-Q}$
\item $PRESS_{(U)} = y_{i} - x\hat{\beta}_{(U)}$
\end{itemize}

\subsection{DFBETA} %1.16.3
\begin{eqnarray}
DFBETA_{a} &=& \hat{\beta} - \hat{\beta}_{(a)} \\
&=& B(Y-Y_{\bar{a}}
\end{eqnarray}
%-------------------------------------------------------------------------------------------------------------------------------------%
\subsection{Influence Statistics for LME models} %1.1.4
Influence statistics can be coarsely grouped by the aspect of estimation that is their primary target:
\begin{itemize}
	\item overall measures compare changes in objective functions: (restricted) likelihood distance (Cook and Weisberg 1982, Ch. 5.2)
	\item influence on parameter estimates: Cook's  (Cook 1977, 1979), MDFFITS (Belsley, Kuh, and Welsch 1980, p. 32)
	\item influence on precision of estimates: CovRatio and CovTrace
	\item influence on fitted and predicted values: PRESS residual, PRESS statistic (Allen 1974), DFFITS (Belsley, Kuh, and Welsch 1980, p. 15)
	\item outlier properties: internally and externally studentized residuals, leverage
\end{itemize}
%---------------------------------------------------------------------------%


\subsection{Variance Ratio} %2.4.2
\begin{itemize}
	\item For fixed effect parameters $\beta$.
\end{itemize}

\subsection{Cook-Weisberg statistic} %2.4.3
\begin{itemize}
	\item For fixed effect parameters $\beta$.
\end{itemize}

\subsection{Andrews-Pregibon statistic} %2.4.4
\begin{itemize}
	\item For fixed effect parameters $\beta$.
\end{itemize}
The Andrews-Pregibon statistic $AP_{i}$ is a measure of influence based on the volume of the confidence ellipsoid.
The larger this statistic is for observation $i$, the stronger the influence that observation will have on the model fit.


%---------------------------------------------------------------------------%

%-------------------------------------------------------------------------------------------------------------------------------------%
\end{document}