\documentclass[Chap5amain.tex]{subfiles}
\begin{document}
%-------------------------------------------------------------------------------------------------%
\section{Leverage and Influence}
\subsection{Influence}
The influence of an observation can be thought of in terms of how much the predicted scores for other observations would differ if the observation in question were not included. 

Cook's D is a good measure of the influence of an observation and is proportional to the sum of the squared differences between predictions made with all observations in the analysis and predictions made leaving out the observation in question. If the predictions are the same with or without the observation in question, then the observation has no influence on the regression model. If the predictions differ greatly when the observation is not included in the analysis, then the observation is influential.

\subsection{Interpreting Cook's Distance}
A common rule of thumb is that an observation with a value of Cook's D over 1.0 has too much influence. As with all rules of thumb, this rule should be applied judiciously and not thoughtlessly.

\subsection{Leverage}
% http://onlinestatbook.com/2/regression/influential.html
% Leverage
The leverage of an observation is based on how much the observation's value on the predictor variable differs from the mean of the predictor variable. The greater an observation's leverage, the more potential it has to be an influential observation. 

For example, an observation with a value equal to the mean on the predictor variable has no influence on the slope of the regression line regardless of its value on the criterion variable. On the other hand, an observation that is extreme on the predictor variable has the potential to affect the slope greatly.

\subsubsection{Calculation of Leverage (h)}
The first step is to standardize the predictor variable so that it has a mean of 0 and a standard deviation of 1. Then, the leverage (h) is computed by squaring the observation's value on the standardized predictor variable, adding 1, and dividing by the number of observations.


\subsection{Summary of Influence Statistics}
\begin{itemize}
\item	\textbf{Studentized Residuals} – Residuals divided by their estimated standard errors (like t-statistics). Observations with values larger than 3 in absolute value are considered outliers.
\item	\textbf{Leverage Values (Hat Diag)} – Measure of how far an observation is from the others in terms of the levels of the independent variables (not the dependent variable). Observations with values larger than $2(k+1)/n$ are considered to be potentially highly influential, where k is the number of predictors and n is the sample size.
\item	\textbf{DFFITS} – Measure of how much an observation has effected its fitted value from the regression model. Values larger than $2\sqrt{(k+1)/n}$ in absolute value are considered highly influential. %Use standardized DFFITS in SPSS.
\item	\textbf{DFBETAS} – Measure of how much an observation has effected the estimate of a regression coefficient (there is one DFBETA for each regression coefficient, including the intercept). Values larger than 2/sqrt(n) in absolute value are considered highly influential.
\\
The measure that measures how much impact each observation has on a particular predictor is DFBETAs The DFBETA for a predictor and for a particular observation is the difference between the regression coefficient calculated for all of the data and the regression coefficient calculated with the observation deleted, scaled by the standard error calculated with the observation deleted. 

\item	\textbf{Cook’s D} – Measure of aggregate impact of each observation on the group of regression coefficients, as well as the group of fitted values. Values larger than 4/n are considered highly influential.
\end{itemize}


\subsection{Influential Observations : DFBeta and DFBetas}
% http://stats.stackexchange.com/questions/22161/how-to-read-cooks-distance-plots
Cook's distance refers to how far, on average, predicted y-values will move if the observation in question is dropped from the data set. dfbeta refers to how much a parameter estimate changes if the observation in question is dropped from the data set. Note that with k covariates, there will be k+1 dfbetas (the intercept,$\beta_0$, and 1 $\beta$ for each covariate). Cook's distance is presumably more important to you if you are doing predictive modeling, whereas dfbeta is more important in explanatory modeling.

\subsection{Cook's Distance}
% Interpretation
% http://stats.stackexchange.com/questions/22161/how-to-read-cooks-distance-plots %
Some texts tell you that points for which Cook's distance is higher than 1 are to be considered as influential. Other texts give you a threshold of $4/N$ or $4/(N−k−1)$, where N is the number of observations and k the number of explanatory variables. In your case the latter formula should yield a threshold around 0.1 .

John Fox (1), in his booklet on regression diagnostics is rather cautious when it comes to giving numerical thresholds. He advises the use of graphics and to examine in closer details the points with "values of D that are substantially larger than the rest". According to Fox, thresholds should just be used to enhance graphical displays.

In your case the observations 7 and 16 could be considered as influential. Well, I would at least have a closer look at them. The observation 29 is not substantially different from a couple of other observations.

(1) Fox, John. (1991). Regression Diagnostics: An Introduction. Sage Publications.

\newpage
\subsection{Leverage}
 leverage is a term used in connection with regression analysis and, in particular, in analyses aimed at identifying those observations that are far away from corresponding average predictor values. Leverage points do not necessarily have a large effect on the outcome of fitting regression models.

Leverage points are those observations, if any, made at extreme or outlying values of the independent variables such that the lack of neighboring observations means that the fitted regression model will pass close to that particular observation.[1]

Modern computer packages for statistical analysis include, as part of their facilities for regression analysis, various quantitative measures for identifying influential observations: among these measures is partial leverage, a measure of how a variable contributes to the leverage of a datum.




%-------------------------------------------------------------- %

\subsection{Residual}
Residual (or error) represents unexplained (or residual) variation after fitting a regression model. It is the difference (or left over) between the observed value of the variable and the value suggested by the regression model.


%------------------------- %

The difference between the observed value of the dependent variable (y) and the predicted value (ŷ) is called the residual (e). Each data point has one residual.

Residual = Observed value - Predicted value 
\[e = y - \hat{y} \]

Both the sum and the mean of the residuals are equal to zero. That is, Σ e = 0 and e = 0.

%--------------------------------- %




\subsection{Introduction}
In statistics and optimization, statistical errors and residuals are two closely related and easily confused measures of the deviation of an observed value of an element of a statistical sample from its "theoretical value". The error (or disturbance) of an observed value is the deviation of the observed value from the (unobservable) true function value, while the residual of an observed value is the difference between the observed value and the estimated function value.

The distinction is most important in regression analysis, where it leads to the concept of studentized residuals.

%----------------------------- %
\subsection{Residual}
A residual (or fitting error), on the other hand, is an observable estimate of the unobservable statistical error. Consider the previous example with men's heights and suppose we have a random sample of n people. The sample mean could serve as a good estimator of the population mean. Then we have:

The difference between the height of each man in the sample and the unobservable population mean is a statistical error, whereas
The difference between the height of each man in the sample and the observable sample mean is a residual.
Note that the sum of the residuals within a random sample is necessarily zero, and thus the residuals are necessarily not independent. The statistical errors on the other hand are independent, and their sum within the random sample is almost surely not zero.

%------------------------------- %

Other uses of the word "error" in statistics: 

The use of the term "error" as discussed in the sections above is in the sense of a deviation of a value from a hypothetical unobserved value. At least two other uses also occur in statistics, both referring to observable prediction errors:

\begin{itemize}
\item Mean square error or mean squared error (abbreviated MSE) and root mean square error (RMSE) refer to the amount by which the values predicted by an estimator differ from the quantities being estimated (typically outside the sample from which the model was estimated).

\item 
Sum of squared errors, typically abbreviated SSE or SSe, refers to the residual sum of squares (the sum of squared residuals) of a regression; this is the sum of the squares of the deviations of the actual values from the predicted values, within the sample used for estimation. Likewise, the sum of absolute errors (SAE) refers to the sum of the absolute values of the residuals, which is minimized in the least absolute deviations approach to regression.

\end{itemize}

%------------------------------- %
\newpage
\subsection{Studentization}
In statistics, a studentized residual is the quotient resulting from the division of a residual by an estimate of its standard deviation. Typically the standard deviations of residuals in a sample vary greatly from one data point to another even when the errors all have the same standard deviation, particularly in regression analysis; thus it does not make sense to compare residuals at different data points without first studentizing. It is a form of a Student's t-statistic, with the estimate of error varying between points.

This is an important technique in the detection of outliers. It is named in honor of William Sealey Gosset, who wrote under the pseudonym Student, and dividing by an estimate of scale is called studentizing, in analogy with standardizing and normalizing: see Studentization.


\subsection{Residual Plots}
A residual plot is a graph that shows the residuals on the vertical axis and the independent variable on the horizontal axis. If the points in a residual plot are randomly dispersed around the horizontal axis, a linear regression model is appropriate for the data; otherwise, a non-linear model is more appropriate.

Below the table on the left shows inputs and outputs from a simple linear regression analysis, and the chart on the right displays the residual (e) and independent variable (X) as a residual plot.

\begin{verbatim}
x	60	70	80	85	95
y	70	65	70	95	85
y.hat	65.411	71.849	78.288	81.507	87.945
e	4.589	-6.849	-8.288	13.493	-2.945
\end{verbatim}


The residual plot shows a fairly random pattern - the first residual is positive, the next two are negative, the fourth is positive, and the last residual is negative. This random pattern indicates that a linear model provides a decent fit to the data.

Below, the residual plots show three typical patterns. The first plot shows a random pattern, indicating a good fit for a linear model. The other plot patterns are non-random (U-shaped and inverted U), suggesting a better fit for a non-linear model.



%--------------------- %

\newpage
\subsection{Key Definitions}
\begin{description}
\item[ Residual] The difference between the predicted value (based on the regression equation) and the actual, observed value.

\item[ Outlier] In linear regression, an outlier is an observation with large residual. In other words, it is an observation whose dependent-variable value is unusual given its value on the predictor variables. An outlier may indicate a sample peculiarity or may indicate a data entry error or other problem.

\item[ Leverage] An observation with an extreme value on a predictor variable is a point with high leverage. Leverage is a measure of how far an independent variable deviates from its mean. High leverage points can have a great amount of effect on the estimate of regression coefficients.

\item[ Influence] An observation is said to be influential if removing the observation substantially changes the estimate of the regression coefficients.  Influence can be thought of as the product of leverage and outlierness.

\item[ Cook's distance ] A measure that combines the information of leverage and residual of the observation.
\end{description}

%--------------------- %
\newpage
\subsection{Leverage}
In statistics, leverage is a term used in connection with regression analysis and, in particular, in analyses aimed at identifying those observations that are far away from corresponding average predictor values. Leverage points do not necessarily have a large effect on the outcome of fitting regression models.

Leverage points are those observations, if any, made at extreme or outlying values of the independent variables such that the lack of neighboring observations means that the fitted regression model will pass close to that particular observation.[1]

Modern computer packages for statistical analysis include, as part of their facilities for regression analysis, various quantitative measures for identifying influential observations: among these measures is partial leverage, a measure of how a variable contributes to the leverage of a datum.

\subsection{Cook's Distance}
In statistics, Cook's Distance or Cook's D is a commonly used estimate of the influence of a data point when performing least squares regression analysis.[1] In a practical ordinary least squares analysis, Cook's distance can be used in several ways: to indicate data points that are particularly worth checking for validity; to indicate regions of the design space where it would be good to be able to obtain more data points. It is named after the American statistician R. Dennis Cook, who introduced the concept in 1977.


\subsubsection{Interpretation}
Specifically $D_i$ can be interpreted as the distance one's estimates move within the confidence ellipsoid that represents a region of plausible values for the parameters.[clarification needed] This is shown by an alternative but equivalent representation of Cook's distance in terms of changes to the estimates of the regression parameters between the cases where the particular observation is either included or excluded from the regression analysis.


%--------------------- %
\newpage
\section{Multivariate}
\subsection{Mahalanobis Distance }

The Mahalanobis Distance is a descriptive statistic that provides a relative measure of a data point's distance (residual) from a common point. It is a unitless measure introduced by P. C. Mahalanobis in 1936.[1] The Mahalanobis distance is used to identify and gauge similarity of an unknown sample set to a known one. It differs from Euclidean distance in that it takes into account the correlations of the data set and is scale-invariant. In other words, it has a multivariate effect size.

\end{document}
