% Check This Document for Duplication

\documentclass[Chap5dmain.tex]{subfiles}
\begin{document}


% \chapter{Augmented GLMs}





\section{Hierarchical likelihood} %3.3
Inferential method was developed for the mixed linear model via Lee and Nelder's (1996) hierarchical-likelihood (h-likelihood).

\section{Importance-Weighted Least-Squares (IWLS)}  %3.4



%-------------------------------------------------------------------------------------------------------------------------------------%
%-------------------------------------------------------------------------------------------------------------------------------------%
%-------------------------------------------------------------------------------------------------Chapter 4------------------------%
%-------------------------------------------------------------------------------------------------------------------------------------%
%-------------------------------------------------------------------------------------------------------------------------------------%

%---------------------------------------------------------------------------%
% - 3. Augmented GLMS
%---------------------------------------------------------------------------%


% Generalized linear models are a generalization of classical linear  models.

\section{Augmented GLMs} %3.1

With the use of h-likihood, a random effected model of the form can be viewed as an `augmented GLM' with the response varaibkes $(y^t, \phi^t_m)^t$, (with $\mu = E(y)$,$ u = E(\phi)$, $var(y) = \theta V (\mu)$.
The augmented linear predictor is \[\eta_{ma}  = (\eta^t, \eta^t_m)^t) = T\omega. \].

%---------------------------------------------------------------------------%

%Augmented Generalized linear models.
% Youngjo et al page 154

The subscript $M$ is a label referring to the mean model.
\begin{equation}
\left(%
\begin{array}{c}
  Y \\
  \psi_{M} \\
\end{array}%
\right) = \left(
\begin{array}{cc}
  X & Z \\
  0 & I \\
\end{array}\right) \left(%
\begin{array}{c}
  \beta \\
  \nu \\
\end{array}%
\right)+ e^{*}
\end{equation}


%Augmented Generalized linear models.


The error term $e^{*}$ is normal with mean zero. The variance matrix of the error term is given by
\begin{equation}
\Sigma_{a} = \left(%
\begin{array}{cc}
  \Sigma & 0 \\
  0 & D \\
\end{array}%
\right).
\end{equation}


y_{a} = T \delta + e^{*}
\end{equation}

Weighted least squares equation


% Youngjo et al page 154


\subsection{The Augmented Model Matrix}  %3.2
\begin{equation}
X = \left(%
\begin{array}{cc}
  T & Z \\
  0 & I \\
\end{array}%
\right)
\delta = \left(%
\begin{array}{c}
  \beta  \\
  \nu  \\
\end{array}%
\right)
\end{equation}



%---------------------------------------------------------------------------%
% - 3. Augmented GLMS
%---------------------------------------------------------------------------%

% 3.1 Intro to Augmented GLMs
% 3.2 The LME as a GLM
% 3.3 Hierarchical likelihood
% 3.4 IWLS


Generalized linear models are a generalization of classical linear  models.


\section{Augmented GLMs} %3.1


With the use of h-likelihood, a random effected model of the form can be viewed as an `augmented GLM' with the response variables $(y^t, \phi^t_m)^t$, (with $\mu = E(y)$,$ u = E(\phi)$, $var(y) = \theta V (\mu)$.
The augmented linear predictor is \[\eta_{ma}  = (\eta^t, \eta^t_m)^t) = T\omega. \].






%Augmented Generalized linear models.
% Youngjo et al page 154


The subscript $M$ is a label referring to the mean model.
\begin{equation}
\left(%
\begin{array}{c}
  Y \\
  \psi_{M} \\
\end{array}%
\right) = \left(
\begin{array}{cc}
  X & Z \\
  0 & I \\
\end{array}\right) \left(%
\begin{array}{c}
  \beta \\
  \nu \\
\end{array}%
\right)+ e^{*}
\end{equation}




%Augmented Generalized linear models.




The error term $e^{*}$ is normal with mean zero. The variance matrix of the error term is given by
\begin{equation}
\Sigma_{a} = \left(%
\begin{array}{cc}
  \Sigma & 0 \\
  0 & D \\
\end{array}%
\right).
\end{equation}


$y_{a} = T \delta + e^{*}$


Weighted least squares equation




% Youngjo et al page 154




\subsection{The Augmented Model Matrix}  %3.1.2
\begin{equation}
X = \left(%
\begin{array}{cc}
  T & Z \\
  0 & I \\
\end{array}%
\right)
\delta = \left(%
\begin{array}{c}
  \beta  \\
  \nu  \\
\end{array}%
\right)
\end{equation}

%-----------------------------------------------------------------------------------%


\section{The LME model as a general linear model} %3.2
Henderson's equations in %(\ref{Henderson:Equations})
can be rewritten $( T^\prime W^{-1} T ) \delta = T^\prime W^{-1} y_{a} $ using
\[
\delta = \pmatrix{\beta \cr b},
\ y_{a} = \pmatrix{
  y \cr \psi
  },
\ T = \pmatrix{
  X & Z  \cr
  0 & I
  },
\ \textrm{and} \ W = \pmatrix{
  \Sigma & 0  \cr
  0 &  D },
\]
where \textbf{cite[Lee:Neld:Pawi:2006]} describe $\psi = 0$ as quasi-data with mean $\mathrm{E}(\psi) = b.$ Their formulation suggests that the joint estimation of the coefficients $\beta$ and $b$ of the linear mixed effects model can be derived via a classical augmented general linear model $y_{a} = T\delta + \varepsilon$ where $\mathrm{E}(\varepsilon) = 0$ and $\mathrm{var}(\varepsilon) = W,$ with \emph{both} $\beta$ and $b$ appearing as fixed parameters. The usefulness of this reformulation of an LME as a general linear model will be revisited.




\end{document}
%-----------------------------------------------------------------------------------%
