\documentclass[12pt, a4paper]{report}
\usepackage{epsfig}
\usepackage{subfigure}
%\usepackage{amscd}
\usepackage{amssymb}
\usepackage{amsbsy}
\usepackage{amsthm}
%\usepackage[dvips]{graphicx}
\usepackage{natbib}
\usepackage{subfiles}
\bibliographystyle{chicago}
\usepackage{vmargin}
\usepackage{index}
% left top textwidth textheight headheight
% headsep footheight footskip
\setmargins{3.0cm}{2.5cm}{15.5 cm}{22cm}{0.5cm}{0cm}{1cm}{1cm}
\renewcommand{\baselinestretch}{1.5}
\pagenumbering{arabic}
\theoremstyle{plain}
\newtheorem{theorem}{Theorem}[section]
\newtheorem{corollary}[theorem]{Corollary}
\newtheorem{ill}[theorem]{Example}
\newtheorem{lemma}[theorem]{Lemma}
\newtheorem{proposition}[theorem]{Proposition}
\newtheorem{conjecture}[theorem]{Conjecture}
\newtheorem{axiom}{Axiom}
\theoremstyle{definition}
\newtheorem{definition}{Definition}[section]
\newtheorem{notation}{Notation}
\theoremstyle{remark}
\newtheorem{remark}{Remark}[section]
\newtheorem{example}{Example}[section]
\renewcommand{\thenotation}{}
\renewcommand{\thetable}{\thesection.\arabic{table}}
\renewcommand{\thefigure}{\thesection.\arabic{figure}}
\title{Research notes: linear mixed effects models}
\author{ } \date{ }


\makeindex
\begin{document}
\author{Kevin O'Brien}
\title{November 2011 Version A}


\addcontentsline{toc}{section}{Bibliography}


\tableofcontents \setcounter{tocdepth}{1}


%---------------------------------------------------------------------------%
% - 1. Model Diagnostics
% - 2. Zewotir's paper (including Haslett)
% - 3. Augmented GLMS
% - 4. Applying Diagnostics to MCS
% - 5. Extra Material
%---------------------------------------------------------------------------%

%---------------------------------------------------------------------------%


%---------------------------------------------------------------------------%
\newpage
\section{Residual diagnostics} %1.3
For classical linear models, residual diagnostics are typically implemented as a plot of the observed residuals and the predicted values. A visual inspection for the presence of trends inform the analyst on the validity of distributional assumptions, and to detect outliers and influential observations.



	%--Marginal and Conditional Residuals
	
\subsection{Residuals diagnostics in mixed models}

%schabenberger
The marginal and conditional means in the linear mixed model are
$E[\boldsymbol{Y}] = \boldsymbol{X}\boldsymbol{\beta}$ and
$E[\boldsymbol{Y|\boldsymbol{u}}] = \boldsymbol{X}\boldsymbol{\beta} + \boldsymbol{Z}\boldsymbol{u}$, respectively.

A residual is the difference between an observed quantity and its estimated or predicted value. In the mixed
model you can distinguish marginal residuals $r_m$ and conditional residuals $r_c$. 


\subsection{Marginal and Conditional Residuals}

A marginal residual is the difference between the observed data and the estimated (marginal) mean, $r_{mi} = y_i - x_0^{\prime} \hat{b}$
A conditional residual is the difference between the observed data and the predicted value of the observation,
$r_{ci} = y_i - x_i^{\prime} \hat{b} - z_i^{\prime} \hat{\gamma}$

In linear mixed effects models, diagnostic techniques may consider `conditional' residuals. A conditional residual is the difference between an observed value $y_{i}$ and the conditional predicted value $\hat{y}_{i} $.

\[ \hat{epsilon}_{i} = y_{i} - \hat{y}_{i} = y_{i} - ( X_{i}\hat{beta} + Z_{i}\hat{b}_{i}) \]

However, using conditional residuals for diagnostics presents difficulties, as they tend to be correlated and their variances may be different for different subgroups, which can lead to erroneous conclusions.

%1.5
%http://support.sas.com/documentation/cdl/en/statug/63033/HTML/default/viewer.htm#statug_mixed_sect024.htm












\chapter{Application to Method Comparison Studies} % Chapter 4




%---------------------------------------------------------------------------%
% - 1. Application to MCS
% - 2. Grubbs' Data
% - 3. R implementation
% - 4. Influence measures using R
%---------------------------------------------------------------------------%


\section{Application to MCS} %4.1


Let $\hat{\beta}$ denote the least square estimate of $\beta$
based upon the full set of observations, and let
$\hat{\beta}^{(k)}$ denoted the estimate with the $k^{th}$ case
excluded.




\section{Grubbs' Data} %4.2


For the Grubbs data the $\hat{\beta}$ estimated are
$\hat{\beta}_{0}$ and $\hat{\beta}_{1}$ respectively. Leaving the
fourth case out, i.e. $k=4$ the corresponding estimates are
$\hat{\beta}_{0}^{-4}$ and $\hat{\beta}_{1}^{-4}$




\begin{equation}
Y^{-Q} = \hat{\beta}^{-Q}X^{-Q}
\end{equation}


When considering the regression of case-wise differences and averages, we write $D^{-Q} = \hat{\beta}^{-Q}A^{-Q}$




\newpage


\begin{table}[ht]
\begin{center}
\begin{tabular}{rrrrr}
  \hline
 & F & C & D & A \\
  \hline
1 & 793.80 & 794.60 & -0.80 & 794.20 \\
  2 & 793.10 & 793.90 & -0.80 & 793.50 \\
  3 & 792.40 & 793.20 & -0.80 & 792.80 \\
  4 & 794.00 & 794.00 & 0.00 & 794.00 \\
  5 & 791.40 & 792.20 & -0.80 & 791.80 \\
  6 & 792.40 & 793.10 & -0.70 & 792.75 \\
  7 & 791.70 & 792.40 & -0.70 & 792.05 \\
  8 & 792.30 & 792.80 & -0.50 & 792.55 \\
  9 & 789.60 & 790.20 & -0.60 & 789.90 \\
  10 & 794.40 & 795.00 & -0.60 & 794.70 \\
  11 & 790.90 & 791.60 & -0.70 & 791.25 \\
  12 & 793.50 & 793.80 & -0.30 & 793.65 \\
   \hline
\end{tabular}
\end{center}
\end{table}




\newpage


\begin{equation}
Y^{(k)} = \hat{\beta}^{(k)}X^{(k)}
\end{equation}


Consider two sets of measurements , in this case F and C , with the vectors of case-wise averages $A$ and case-wise differences $D$ respectively. A regression model of differences on averages can be fitted with the view to exploring some characteristics of the data.


When considering the regression of case-wise differences and averages, we write


\begin{equation}
D^{-Q} = \hat{\beta}^{-Q}A^{-Q}
\end{equation}
Let $\hat{\beta}$ denote the least square estimate of $\beta$ based upon the full set of observations, and let $\hat{\beta}^{(k)}$ denoted the estimate with the $k^{th}$ case excluded.


For the Grubbs data the $\hat{\beta}$ estimated are $\hat{\beta}_{0}$ and $\hat{\beta}_{1}$ respectively. Leaving the
fourth case out, i.e. $k=4$ the corresponding estimates are $\hat{\beta}_{0}^{-4}$ and $\hat{\beta}_{1}^{-4}$


\begin{equation}
Y^{(k)} = \hat{\beta}^{(k)}X^{(k)}
\end{equation}


Consider two sets of measurements , in this case F and C , with the vectors of case-wise averages $A$ and case-wise differences $D$ respectively. A regression model of differences on averages can be fitted with the view to exploring some characteristics of the data.


\begin{verbatim}
Call: lm(formula = D ~ A)


Coefficients: (Intercept)            A
  -37.51896      0.04656


\end{verbatim}








When considering the regression of case-wise differences and averages, we write


\begin{equation}
D^{-Q} = \hat{\beta}^{-Q}A^{-Q}
\end{equation}






\subsection{Influence measures using R} %4.4
\texttt{R} provides the following influence measures of each observation.


%Influence measures: This suite of functions can be used to compute
%some of the regression (leave-one-out deletion) diagnostics for
%linear and generalized linear models discussed in Belsley, Kuh and
% Welsch (1980), Cook and Weisberg (1982)






\begin{table}[ht]
\begin{center}
\begin{tabular}{|c|c|c|c|c|c|c|}
  \hline
 & dfb.1\_ & dfb.A & dffit & cov.r & cook.d & hat \\
  \hline
1 & 0.42 & -0.42 & -0.56 & 1.13 & 0.15 & 0.18 \\
  2 & 0.17 & -0.17 & -0.34 & 1.14 & 0.06 & 0.11 \\
  3 & 0.01 & -0.01 & -0.24 & 1.17 & 0.03 & 0.08 \\
  4 & -1.08 & 1.08 & 1.57 & 0.24 & 0.56 & 0.16 \\
  5 & -0.14 & 0.14 & -0.24 & 1.30 & 0.03 & 0.13 \\
  6 & -0.00 & 0.00 & -0.11 & 1.31 & 0.01 & 0.08 \\
  7 & -0.04 & 0.04 & -0.08 & 1.37 & 0.00 & 0.11 \\
  8 & 0.02 & -0.02 & 0.15 & 1.28 & 0.01 & 0.09 \\
  9 & 0.69 & -0.68 & 0.75 & 2.08 & 0.29 & 0.48 \\
  10 & 0.18 & -0.18 & -0.22 & 1.63 & 0.03 & 0.27 \\
  11 & -0.03 & 0.03 & -0.04 & 1.53 & 0.00 & 0.19 \\
  12 & -0.25 & 0.25 & 0.44 & 1.05 & 0.09 & 0.12 \\
   \hline
\end{tabular}
\end{center}
\end{table}






\printindex
\bibliographystyle{chicago}
\bibliography{DB-txfrbib}


\end{document}
