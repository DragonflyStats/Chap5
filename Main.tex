\documentclass[12pt, a4paper]{report}
\usepackage{epsfig}
\usepackage{subfigure}
%\usepackage{amscd}
\usepackage{amssymb}
\usepackage{framed}
\usepackage{subfiles}
\usepackage{amsbsy}
\usepackage{amsthm}
%\usepackage[dvips]{graphicx}
\usepackage{natbib}
\usepackage{subfiles}
\bibliographystyle{chicago}
\usepackage{vmargin}
\usepackage{index}
% left top textwidth textheight headheight
% headsep footheight footskip
\setmargins{3.0cm}{2.5cm}{15.5 cm}{22cm}{0.5cm}{0cm}{1cm}{1cm}
\renewcommand{\baselinestretch}{1.5}
\pagenumbering{arabic}
\theoremstyle{plain}
\newtheorem{theorem}{Theorem}[section]
\newtheorem{corollary}[theorem]{Corollary}
\newtheorem{ill}[theorem]{Example}
\newtheorem{lemma}[theorem]{Lemma}
\newtheorem{proposition}[theorem]{Proposition}
\newtheorem{conjecture}[theorem]{Conjecture}
\newtheorem{axiom}{Axiom}
\theoremstyle{definition}
\newtheorem{definition}{Definition}[section]
\newtheorem{notation}{Notation}
\theoremstyle{remark}
\newtheorem{remark}{Remark}[section]
\newtheorem{example}{Example}[section]
\renewcommand{\thenotation}{}
\renewcommand{\thetable}{\thesection.\arabic{table}}
\renewcommand{\thefigure}{\thesection.\arabic{figure}}
\title{Research notes: linear mixed effects models}
\author{ } \date{ }


\makeindex
\begin{document}
	\author{Kevin O'Brien}
	\title{November 2011 Version A}
	
	
	\addcontentsline{toc}{section}{Bibliography}
	
	
	\tableofcontents \setcounter{tocdepth}{1}
	
\begin{framed} 
\begin{itemize}
	\item \texttt{R} command and \texttt{R} object - Typewriter Font
	\item \texttt{R} Package name - Italics
	\item Selected Acronyms and Proper Nouns - Italics
\end{itemize}
\end{framed}
\newpage


This chapter is broken into two parts. The first part is a review of diagnostics methods for linear models, intended to acquaint the
reader with the subject, and also to provide a basis for material covered in the second part. Particular attention is drawn to graphical methods.

The second part of the chapter looks at diagnostics techniques for LME models, firsly covering the theory, then proceeding to a discussion on 
implementing these using \texttt{R} code.
While a substantial body of work has been developed in this area, ther are still area worth exploring. 
In particular the development of graphical techniques pertinent to LME models should be looked at.
\newpage


\subfile{1.1-InfluenceIntro.tex}
\subfile{1.1-ResidualPlots.tex}
\subfile{1.2-Studentization.tex}
\subfile{1.2-CondMargResidual.tex}
\subfile{1.2-Influence-OLS.tex}
\subfile{1.3-Influence1.tex}
\subfile{1.4-InfluenceAnalysis.tex}
\subfile{1.6-Influence-LME.tex}
\subfile{1.7-MeasureOfInfluence.tex}
\subfile{1.8-Schabenberger.tex}
\subfile{2.1-Zewotir.tex}
\subfile{2.3-LindMun-LMEdiag.tex}
\subfile{2.4-CaseDeletionDiagnostics.tex}
\subfile{2.5-CooksDistance-OLS.tex}
\subfile{2.7-InfluenceCooksDistance.tex}
\subfile{2.8-CooksDistance-LME.tex}
\subfile{2.6-CPJ.tex}
\subfile{3.1-LikelihoodDistances.tex}
\subfile{3.2-Leverage.tex}
\subfile{3.3-NobreSinger-LMELeverage.tex}
\subfile{5.2-HaslettDillane.tex}
\subfile{5.3-HaslettHayes.tex}
\subfile{4.1-Turkan-InfluenceAnalysis.tex}
\subfile{6.3-Influence.ME-Rpackage.tex}

%===================================================================%
\newpage
\begin{itemize}
\item \textit{
	The previous Section (Section 4) is a literary review of residual diagnostics and influence procedures
	for Linear Mixed Effects Models, drawing heavily on Schabenberger and Zewotir.}
	
\item \textit{	Section 4 begins with an introduction to key topics in residual diagnostics, such as influence, leverage, outliers
	and Cook's distance. Other concepts such as DFFITS and DFBETAs will be introduced briefly, mostly to explain why the are not particularly useful for
	the Method Comparison context, and therefore are not elaborated upon.}
	
\item \textit{	In brief, Variable Selection is not applicable to Method Comparison Studies, in the 
	commonly used used context. 
	Testing a rather simplisticy specificied model against one with more random effects terms is tractable, but this research question is of secondary importance.}
\end{itemize}

%=============================================== %
\newpage
\subsection*{Appendix to Section 4}

As an appendix to section 4, an appraisal of the current state of development (or lack thereof) for current implemenations for LME models, particularly for
\texttt{nlme} and \texttt{lme4} fitted models.

Crucially, a review of internet resources indicates that almost all of the progress in this regard has been done for \texttt{lme4} fitted models, specifically the \textit{Influence.ME} \texttt{R} package. (Nieuwenhuis et 2012)

Conversely there is very little for \texttt{nlme} models. To delve into this mor, one would immediately investigate the current development workflow for both packages.

%======================%
% Douglas Bates

As an aside, Douglas Bates was arguably the most prominent \texttt{R} developer working in the LME area. 
However Bates has now prioritised the development of LME models in another computing environment , i.e Julia. 
% The current version of this is XXXX

%======================%
% nlme
\subsubsection*{The \texttt{nlme} package}

With regards to \texttt{nlme}, the torch has been passed to Galecki Galecki \& Burzykowski (UMich. and Hasselt respecitely).  Galecki \& Burzykowski published \textit{Linear Mixed Effects Models using \texttt{R}}. 
Also, the accompanying \texttt{R} package, nlmeU package is under current development, with a version being released XXXX.


%======================%
% lme4 and influence.ME
\subsubsection*{The \texttt{lme4} package}

The \texttt{lme4} package is also under active development, under the leadership of Ben Bolker (McMaster University). According to CRAN, the LME4 package, fits linear and generalized linear mixed-effects models

\begin{quote}
	The models and their components are represented using S4 classes and methods. The core computational algorithms are implemented using the Eigen C++ library for numerical linear algebra and RcppEigen "glue".
	(CRAN)
\end{quote}

%=====================%
% Important Consideration for MCS

The key issue is that \texttt{nlme} allows for the particular specification of Roy's Model, speciifically direct spefiication of the VC matrices for within subject and between subject residuals.
The \texttt{lme4} package does not allow for this.
To advance the ideas that eminate from Roys' paper, one is required to use the \texttt{nlme} context. However, to take advantage of the infrastructure already provided for \texttt{lme4} models, one may change the research question away from that of Roy's paper. 
To this end, an exploration of what textit{influence.ME} can accomplished is merited.
As a complement to this, one can also consider how to properly employ the $R^2$ measure, in the context of Methoc Comparison Studies, further to the work by Edwards et al, namely ``An $R^2$ statistic for fixed effects in the linear mixed model".
%================================================= %
\newpage
\begin{framed}

	\begin{quote}
		\textbf{Abstract for ``An $R^2$ statistic for fixed effects in the linear mixed model"}
		Statisticians most often use the linear mixed model to analyze Gaussian longitudinal data. 
		
		The value and familiarity of the R2 statistic in the linear univariate model naturally creates great interest in extending it to the linear mixed model. We define and describe how to compute a model R2 statistic for the linear mixed model by using only a single model. 
		
		The proposed R2 statistic measures multivariate association between the repeated outcomes and the fixed effects in the linear mixed model. The R2 statistic arises as a 1–1 function of an appropriate F statistic for testing all fixed effects (except typically the intercept) in a full model. 
		
		The statistic compares the full model with a null model with all fixed effects deleted (except typically the intercept) while retaining exactly the same covariance structure. 
		
		Furthermore, the R2 statistic leads immediately to a natural definition of a partial R2 statistic. A mixed model in which ethnicity gives a very small p-value as a longitudinal predictor of blood pressure (BP) compellingly illustrates the value of the statistic. 
		
		In sharp contrast to the extreme p-value, a very small $R^2$ , a measure of statistical and scientific importance, indicates that ethnicity has an almost negligible association with the repeated BP outcomes for the study.
	\end{quote}
\end{framed}
%====================%
% Diagnostics with nlmeU
\newpage
\subsection*{Leave-One-Out Diagnostics with \texttt{lmeU}}
Galecki et al discuss the matter of LME influence diagnostics in their book, although not into great detail.


The command \texttt{lmeU} fits a model with a particular subject removed. The identifier of the subject to be removed is passed as the only argument

A plot ofthe per-observation diagnostics individual subject log-likelihood contributions can be rendered.

\subsubsection*{Likelihood Displacement}
%% Page 503 Galecki




%====================================================================%
\newpage
%\subfile{LikelihoodDistances.tex}


\newpage
\subsection*{Missing Data in Method Comparison Studies}

The matter of missing data has not been commonly encountered in either Method Comparison Studies or Linear Mixed Effects Modelling. However Roy (2009) deals with the relevant assumptions regrading missing data.

Galecki \& Burzykowski (2013) tackles the subject of missing data in LME Modelling.

Furthermore the nlmeU package includes the \texttt{patMiss} function, which ``allows to compactly present pattern of missing data in a given vector/matrix/data
frame or combination of thereof".

\newpage
\bibliography{DB-txfrbib}
\end{document}


