3.2 Leverage

Leverage

For the general mixed model, leverage can be defined through the projection matrix that results from a transformation of the model with the inverse of the Cholesky decomposition of , or through an oblique projector. The MIXED procedure follows the latter path in the computation of influence diagnostics. The leverage value reported for the th observation is the th diagonal entry of the matrix

which is the weight of the observation in contributing to its own predicted value, .
While  is idempotent, it is generally not symmetric and thus not a projection matrix in the narrow sense.
The properties of these leverages are generalizations of the properties in models with diagonal variance-covariance matrices. For example, , and in a model with intercept and , the leverage values

are  and . The lower bound for  is achieved in an intercept-only model, and the upper bound is achieved in a saturated model. The trace of  equals the rank of .
If  denotes the element in row , column  of , then for a model containing only an intercept the diagonal elements of  are

Because  is a sum of elements in the th row of the inverse variance-covariance matrix,  can be negative, even if the correlations among data points are nonnegative. In case of a saturated model with , .