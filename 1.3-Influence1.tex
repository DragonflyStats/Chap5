\documentclass[Chap5amain.tex]{subfiles}
\begin{document}
%---------------------------------------------------------------------------%
%1.1 Introduction to Influence Analysis
%1.2 Extension of techniques to LME Models
%1.3 Residual Diagnostics
%1.4 Standardized and studentized residuals
%1.5 Covariance Parameters
%1.6 Case Deletion Diagnostics
%1.7 Influence Analysis
%1.8 Terminology for Case Deletion
%1.9 Cook's Distance (Classical Case)
%1.10 Cook's Distance (LME Case)
%1.11 Likelihood Distance
%1.12 Other Measures
%1.13 CPJ Paper
%1.14 Matrix Notation of Case Deletion
%1.15 CPJ's Three Propositions
%1.16 Other measures of Influence
%---------------------------------------------------------------------------%
%-------------------------------------------------------------------------------------------------%
\section{Leverage and Influence}
\subsection{Influence}
The influence of an observation can be thought of in terms of how much the predicted scores for other observations would differ if the observation in question were not included. 

Cook's D is a good measure of the influence of an observation and is proportional to the sum of the squared differences between predictions made with all observations in the analysis and predictions made leaving out the observation in question. If the predictions are the same with or without the observation in question, then the observation has no influence on the regression model. If the predictions differ greatly when the observation is not included in the analysis, then the observation is influential.

\subsection{Interpreting Cook's Distance}
A common rule of thumb is that an observation with a value of Cook's D over 1.0 has too much influence. As with all rules of thumb, this rule should be applied judiciously and not thoughtlessly.

\subsection{Leverage}
% http://onlinestatbook.com/2/regression/influential.html
% Leverage
The leverage of an observation is based on how much the observation's value on the predictor variable differs from the mean of the predictor variable. The greater an observation's leverage, the more potential it has to be an influential observation. 

For example, an observation with a value equal to the mean on the predictor variable has no influence on the slope of the regression line regardless of its value on the criterion variable. On the other hand, an observation that is extreme on the predictor variable has the potential to affect the slope greatly.

\subsubsection{Calculation of Leverage (h)}
The first step is to standardize the predictor variable so that it has a mean of 0 and a standard deviation of 1. Then, the leverage (h) is computed by squaring the observation's value on the standardized predictor variable, adding 1, and dividing by the number of observations.


\subsection{Summary of Influence Statistics}
\begin{itemize}
	\item	\textbf{Studentized Residuals} – Residuals divided by their estimated standard errors (like t-statistics). Observations with values larger than 3 in absolute value are considered outliers.
	\item	\textbf{Leverage Values (Hat Diag)} – Measure of how far an observation is from the others in terms of the levels of the independent variables (not the dependent variable). Observations with values larger than $2(k+1)/n$ are considered to be potentially highly influential, where k is the number of predictors and n is the sample size.
	\item	\textbf{DFFITS} – Measure of how much an observation has effected its fitted value from the regression model. Values larger than $2\sqrt{(k+1)/n}$ in absolute value are considered highly influential. %Use standardized DFFITS in SPSS.
	\item	\textbf{DFBETAS} – Measure of how much an observation has effected the estimate of a regression coefficient (there is one DFBETA for each regression coefficient, including the intercept). Values larger than 2/sqrt(n) in absolute value are considered highly influential.
	\\
	The measure that measures how much impact each observation has on a particular predictor is DFBETAs The DFBETA for a predictor and for a particular observation is the difference between the regression coefficient calculated for all of the data and the regression coefficient calculated with the observation deleted, scaled by the standard error calculated with the observation deleted. 
	
	\item	\textbf{Cook’s D} – Measure of aggregate impact of each observation on the group of regression coefficients, as well as the group of fitted values. Values larger than 4/n are considered highly influential.
\end{itemize}
\newpage
\section{Introduction}%1.1
In classical linear models model diagnostics have been become a required part of any statistical analysis, and the methods are commonly available in statistical packages and standard textbooks on applied regression. However it has been noted by several papers that model diagnostics do not often accompany LME model analyses.
Model diagnostic techniques determine whether or not the distributional assumptions are satisfied, and to assess the influence of unusual observations.


\subsection{Model Data Agreement} %1.1.1
\citet{schabenberger} describes the examination of model-data agreement as comprising several elements; residual analysis, goodness of fit, collinearity diagnostics and influence analysis.


\subsection{Influence Diagnostics: Basic Idea and Statistics} %1.1.2
%http://support.sas.com/documentation/cdl/en/statug/63033/HTML/default/viewer.htm#statug_mixed_sect024.htm


The general idea of quantifying the influence of one or more observations relies on computing parameter estimates based on all data points, removing the cases in question from the data, refitting the model, and computing statistics based on the change between full-data and reduced-data estimation.






\subsection{Influence Analysis for LME Models} %1.1.3
The linear mixed effects model is a useful methodology for fitting a wide range of models. However, linear mixed effects models are known to be sensitive to outliers. \citet{Christensen} advises that identification of outliers is necessary before conclusions may be drawn from the fitted model.


Standard statistical packages concentrate on calculating and testing parameter estimates without considering the diagnostics of the model.The assessment of the effects of perturbations in data, on the outcome of the analysis, is known as statistical influence analysis. Influence analysis examines the robustness of the model. Influence analysis methodologies have been used extensively in classical linear models, and provided the basis for methodologies for use with LME models.
Computationally inexpensive diagnostics tools have been developed to examine the issue of influence \citep{Zewotir}.
Studentized residuals, error contrast matrices and the inverse of the response variance covariance matrix are regular components of these tools.


\subsection{Influence Statistics for LME models} %1.1.4
Influence statistics can be coarsely grouped by the aspect of estimation that is their primary target:
\begin{itemize}
\item overall measures compare changes in objective functions: (restricted) likelihood distance (Cook and Weisberg 1982, Ch. 5.2)
\item influence on parameter estimates: Cook's  (Cook 1977, 1979), MDFFITS (Belsley, Kuh, and Welsch 1980, p. 32)
\item influence on precision of estimates: CovRatio and CovTrace
\item influence on fitted and predicted values: PRESS residual, PRESS statistic (Allen 1974), DFFITS (Belsley, Kuh, and Welsch 1980, p. 15)
\item outlier properties: internally and externally studentized residuals, leverage
\end{itemize}
%---------------------------------------------------------------------------%




\subsection{What is Influence} %1.1.5


Broadly defined, influence is understood as the ability of a single or multiple data points, through their presence or absence in the data, to alter important aspects of the analysis, yield qualitatively different inferences, or violate assumptions of the statistical model. The goal of influence analysis is not primarily to mark data
points for deletion so that a better model fit can be achieved for the reduced data, although this might be a result of influence analysis \citep{schabenberger}.


%-------%
\subsection{Quantifying Influence}  %1.1.6


The basic procedure for quantifying influence is simple as follows:


\begin{itemize}
\item Fit the model to the data and obtain estimates of all parameters.
\item Remove one or more data points from the analysis and compute updated estimates of model parameters.
\item Based on full- and reduced-data estimates, contrast quantities of interest to determine how the absence of the observations changes the analysis.
\end{itemize}


\citet{cook86} introduces powerful tools for local-influence assessment and examining perturbations in the assumptions of a model. In particular the effect of local perturbations of parameters or observations are examined.




%---------------------------------------------------------------------------%
\newpage
\section{Extension of techniques to LME Models} %1.2


Model diagnostic techniques, well established for classical models, have since been adapted for use with linear mixed effects models.Diagnostic techniques for LME models are inevitably more difficult to implement, due to the increased complexity.


Beckman, Nachtsheim and Cook (1987) \citet{Beckman} applied the \index{local influence}local influence method of Cook (1986) to the analysis of the linear mixed model.


While the concept of influence analysis is straightforward, implementation in mixed models is more complex. Update formulae for fixed effects models are available only when the covariance parameters are assumed to be known.


If the global measure suggests that the points in $U$ are influential, the nature of that influence should be determined. In particular, the points in $U$ can affect the following


\begin{itemize}
\item the estimates of fixed effects,
\item the estimates of the precision of the fixed effects,
\item the estimates of the covariance parameters,
\item the estimates of the precision of the covariance parameters,
\item fitted and predicted values.
\end{itemize}




%---------------------------------------------------------------------------%
\newpage
\section{Residual diagnostics} %1.3
For classical linear models, residual diagnostics are typically implemented as a plot of the observed residuals and the predicted values. A visual inspection for the presence of trends inform the analyst on the validity of distributional assumptions, and to detect outliers and influential observations.






    %--Marginal and Conditional Residuals


\subsection{Residuals diagnostics in mixed models}


%schabenberger
The marginal and conditional means in the linear mixed model are
$E[\boldsymbol{Y}] = \boldsymbol{X}\boldsymbol{\beta}$ and
$E[\boldsymbol{Y|\boldsymbol{u}}] = \boldsymbol{X}\boldsymbol{\beta} + \boldsymbol{Z}\boldsymbol{u}$, respectively.


A residual is the difference between an observed quantity and its estimated or predicted value. In the mixed
model you can distinguish marginal residuals $r_m$ and conditional residuals $r_c$.




\subsection{Marginal and Conditional Residuals}


A marginal residual is the difference between the observed data and the estimated (marginal) mean, $r_{mi} = y_i - x_0^{\prime} \hat{b}$
A conditional residual is the difference between the observed data and the predicted value of the observation,
$r_{ci} = y_i - x_i^{\prime} \hat{b} - z_i^{\prime} \hat{\gamma}$


In linear mixed effects models, diagnostic techniques may consider `conditional' residuals. A conditional residual is the difference between an observed value $y_{i}$ and the conditional predicted value $\hat{y}_{i} $.


\[ \hat{epsilon}_{i} = y_{i} - \hat{y}_{i} = y_{i} - ( X_{i}\hat{beta} + Z_{i}\hat{b}_{i}) \]


However, using conditional residuals for diagnostics presents difficulties, as they tend to be correlated and their variances may be different for different subgroups, which can lead to erroneous conclusions.


%1.5
%http://support.sas.com/documentation/cdl/en/statug/63033/HTML/default/viewer.htm#statug_mixed_sect024.htm












\begin{equation}
r_{mi}=x^{T}_{i}\hat{\beta}
\end{equation}


\subsection{Marginal Residuals}
\begin{eqnarray}
\hat{\beta} &=& (X^{T}R^{-1}X)^{-1}X^{T}R^{-1}Y \nonumber \\
&=& BY \nonumber
\end{eqnarray}




%---------------------------------------------------------------------------%
\newpage
\section{Standardized and studentized residuals} %1.4
    %--Studentized and Standardized Residuals


To alleviate the problem caused by inconstant variance, the residuals are scaled (i.e. divided) by their standard deviations. This results in a \index{standardized residual}`standardized residual'. Because true standard deviations are frequently unknown, one can instead divide a residual by the estimated standard deviation to obtain the \index{studentized residual}`studentized residual.


\subsection{Standardization} %1.4.1


A random variable is said to be standardized if the difference from its mean is scaled by its standard deviation. The residuals above have mean zero but their variance is unknown, it depends on the true values of $\theta$. Standardization is thus not possible in practice.


\subsection{Studentization} %1.4.2
Instead, you can compute studentized residuals by dividing a residual by an estimate of its standard deviation.


\subsection{Internal and External Studentization} %1.4.3
If that estimate is independent of the $i-$th observation, the process is termed \index{external studentization}`external studentization'. This is usually accomplished by excluding the $i-$th observation when computing the estimate of its standard error. If the observation contributes to the
standard error computation, the residual is said to be \index{internally studentization}internally studentized.


Externally \index{studentized residual} studentized residual require iterative influence analysis or a profiled residuals variance.




\subsection{Computation}%1.4.4


The computation of internally studentized residuals relies on the diagonal entries of $\boldsymbol{V} (\hat{\theta})$ - $\boldsymbol{Q} (\hat{\theta})$, where $\boldsymbol{Q} (\hat{\theta})$ is computed as


\[ \boldsymbol{Q} (\hat{\theta}) = \boldsymbol{X} ( \boldsymbol{X}^{\prime}\boldsymbol{Q} (\hat{\theta})^{-1}\boldsymbol{X})\boldsymbol{X}^{-1} \]


\subsection{Pearson Residual}%1.4.5


Another possible scaled residual is the \index{Pearson residual} `Pearson residual', whereby a residual is divided by the standard deviation of the dependent variable. The Pearson residual can be used when the variability of $\hat{\beta}$ is disregarded in the underlying assumptions.


%---------------------------------------------------------------------------%
\newpage
\section{Covariance Parameters} %1.5
The unknown variance elements are referred to as the covariance parameters and collected in the vector $\theta$.
% - where is this coming from?
% - where is it used again?
% - Has this got anything to do with CovTrace etc?
%---------------------------------------------------------------------------%
\newpage
\section{Case Deletion Diagnostics} %1.6


\citet{CPJ} develops \index{case deletion diagnostics} case deletion diagnostics, in particular the equivalent of \index{Cook's distance} Cook's distance, for diagnosing influential observations when estimating the fixed effect parameters and variance components.


\subsection{Deletion Diagnostics}


Since the pioneering work of Cook in 1977, deletion measures have been applied to many statistical models for identifying influential observations.


Deletion diagnostics provide a means of assessing the influence of an observation (or groups of observations) on inference on the estimated parameters of LME models.


Data from single individuals, or a small group of subjects may influence non-linear mixed effects model selection. Diagnostics routinely applied in model building may identify such individuals, but these methods are not specifically designed for that purpose and are, therefore, not optimal. We describe two likelihood-based diagnostics for identifying individuals that can influence the choice between two competing models.


Case-deletion diagnostics provide a useful tool for identifying influential observations and outliers.


The computation of case deletion diagnostics in the classical model is made simple by the fact that estimates of $\beta$ and $\sigma^2$, which exclude the ith observation, can be computed without re-fitting the model. Such update formulas are available in the mixed model only if you assume that the covariance parameters are not affected by the removal of the observation in question. This is rarely a reasonable assumption.


\subsection{Effects on fitted and predicted values}
\begin{equation}
\hat{e_{i}}_{(U)} = y_{i} - x\hat{\beta}_{(U)}
\end{equation}






A general method for comparing nested models fit by maximum liklihood is the liklihood ratio 
test. This test can be used for models fit by REML (restricted maximum liklihood), but only if the 
fixed terms in the two models are invariant, and both models have been fit by REML. Otherwise, 
the argument: method=”ML” must be employed (ML = maximum liklihood). 
 
Example of a liklihood ratio test used to compare two models: 
 
!"%;=0%1&=*#(?5"&=*#(B8"
 
The output will contain a p-value, and this should be used in conjunction with the AIC scores to 
judge which model is preferred. Lower AIC scores are better. 
 
Generally, liklihood ratio tests should be used to evaluate the significance of terms on the 
random effects portion of two nested models, and should not be used to determine the 
significance of the fixed effects. 
 
A simple way to more reliably test for the significance of fixed effects in an LME model is to use 
conditional F-tests, as implemented with the simple “anova” function. 
 
Example: 
"
!"%;=0%1&=*#(?8"
 
will give the most reliable test of the fixed effects included in model1. 





\subsection{Methods and Measures}
The key to making deletion diagnostics useable is the development of efficient computational formulas, allowing one to obtain the \index{case deletion diagnostics} case deletion diagnostics by making use of basic building blocks, computed only once for the full model.


\citet{Zewotir} lists several established methods of analyzing influence in LME models. These methods include \begin{itemize}
\item Cook's distance for LME models,
\item \index{likelihood distance} likelihood distance,
\item the variance (information) ration,
\item the \index{Cook-Weisberg statistic} Cook-Weisberg statistic,
\item the \index{Andrews-Prebigon statistic} Andrews-Prebigon statistic.
\end{itemize}






%--------------------------------------------------------------------------%
\newpage
\section{Terminology for Case Deletion diagnostics} %1.8


\citet{preisser} describes two type of diagnostics. When the set consists of only one observation, the type is called
'observation-diagnostics'. For multiple observations, Preisser describes the diagnostics as 'cluster-deletion' diagnostics.






%---------------------------------------------------------------------------%
\newpage
\newpage
\section{Likelihood Distance} %1.11
The likelihood distance gives the amount by which the log-likelihood of the full data changes if one were
to evaluate it at the reduced-data estimates. The important point is that $l(\psi_{(U)})$ is not the log-likelihood
obtained by fitting the model to the reduced data set.


It is obtained by evaluating the likelihood function based on the full data set (containing all n observations) at the reduced-data estimates.


The likelihood distance is a global, summary measure, expressing the joint influence of the observations in
the set $U$ on all parameters in $\psi$  that were subject to updating.
%------------%


\subsection{Likelihood Distance}


The \index{likelihood distance} likelihood distance is a global, summary measure, expressing the joint influence of the observations in the set $U$ on all parameters in $\phi$  that were subject to updating.








%---------------------------------------------------------------------------%
\newpage
\section{Iterative and non-iterative influence analysis} %1.13
\citet{schabenberger} highlights some of the issue regarding implementing mixed model diagnostics.


A measure of total influence requires updates of all model parameters.


however, this doesnt increase the procedures execution time by the same degree.
\subsection{Iterative Influence Analysis}


%----schabenberger page 8
For linear models, the implementation of influence analysis is straightforward.
However, for LME models, the process is more complex. Update formulas for the fixed effects are available only when the covariance parameters are assumed to be known. A measure of total influence requires updates of all model parameters.
This can only be achieved in general is by omitting observations, then refitting the model.


\citet{schabenberger} describes the choice between \index{iterative influence analysis} iterative influence analysis and \index{non-iterative influence analysis} non-iterative influence analysis.




%---------------------------------------------------------------------------%
\newpage
\section{Residual diagnostics} %1.3
For classical linear models, residual diagnostics are typically implemented as a plot of the observed residuals and the predicted values. A visual inspection for the presence of trends inform the analyst on the validity of distributional assumptions, and to detect outliers and influential observations.



	%--Marginal and Conditional Residuals
	
\subsection{Residuals diagnostics in mixed models}

%schabenberger
The marginal and conditional means in the linear mixed model are
$E[\boldsymbol{Y}] = \boldsymbol{X}\boldsymbol{\beta}$ and
$E[\boldsymbol{Y|\boldsymbol{u}}] = \boldsymbol{X}\boldsymbol{\beta} + \boldsymbol{Z}\boldsymbol{u}$, respectively.

A residual is the difference between an observed quantity and its estimated or predicted value. In the mixed
model you can distinguish marginal residuals $r_m$ and conditional residuals $r_c$. 


\subsection{Marginal and Conditional Residuals}

A marginal residual is the difference between the observed data and the estimated (marginal) mean, $r_{mi} = y_i - x_0^{\prime} \hat{b}$
A conditional residual is the difference between the observed data and the predicted value of the observation,
$r_{ci} = y_i - x_i^{\prime} \hat{b} - z_i^{\prime} \hat{\gamma}$

In linear mixed effects models, diagnostic techniques may consider `conditional' residuals. A conditional residual is the difference between an observed value $y_{i}$ and the conditional predicted value $\hat{y}_{i} $.

\[ \hat{epsilon}_{i} = y_{i} - \hat{y}_{i} = y_{i} - ( X_{i}\hat{beta} + Z_{i}\hat{b}_{i}) \]

However, using conditional residuals for diagnostics presents difficulties, as they tend to be correlated and their variances may be different for different subgroups, which can lead to erroneous conclusions.

%1.5
%http://support.sas.com/documentation/cdl/en/statug/63033/HTML/default/viewer.htm#statug_mixed_sect024.htm






\begin{equation}
r_{mi}=x^{T}_{i}\hat{\beta}
\end{equation}

\subsection{Marginal Residuals}
\begin{eqnarray}
\hat{\beta} &=& (X^{T}R^{-1}X)^{-1}X^{T}R^{-1}Y \nonumber \\
&=& BY \nonumber
\end{eqnarray}


%---------------------------------------------------------------------------%
\newpage
\section{Standardized and studentized residuals} %1.4
	%--Studentized and Standardized Residuals

To alleviate the problem caused by inconstant variance, the residuals are scaled (i.e. divided) by their standard deviations. This results in a \index{standardized residual}`standardized residual'. Because true standard deviations are frequently unknown, one can instead divide a residual by the estimated standard deviation to obtain the \index{studentized residual}`studentized residual. 

\subsection{Standardization} %1.4.1

A random variable is said to be standardized if the difference from its mean is scaled by its standard deviation. The residuals above have mean zero but their variance is unknown, it depends on the true values of $\theta$. Standardization is thus not possible in practice.

\subsection{Studentization} %1.4.2
Instead, you can compute studentized residuals by dividing a residual by an estimate of its standard deviation. 

\subsection{Internal and External Studentization} %1.4.3
If that estimate is independent of the $i-$th observation, the process is termed \index{external studentization}`external studentization'. This is usually accomplished by excluding the $i-$th observation when computing the estimate of its standard error. If the observation contributes to the
standard error computation, the residual is said to be \index{internally studentization}internally studentized.

Externally \index{studentized residual} studentized residual require iterative influence analysis or a profiled residuals variance.


\subsection{Computation}%1.4.4

The computation of internally studentized residuals relies on the diagonal entries of $\boldsymbol{V} (\hat{\theta})$ - $\boldsymbol{Q} (\hat{\theta})$, where $\boldsymbol{Q} (\hat{\theta})$ is computed as

\[ \boldsymbol{Q} (\hat{\theta}) = \boldsymbol{X} ( \boldsymbol{X}^{\prime}\boldsymbol{Q} (\hat{\theta})^{-1}\boldsymbol{X})\boldsymbol{X}^{-1} \]

\subsection{Pearson Residual}%1.4.5

Another possible scaled residual is the \index{Pearson residual} `Pearson residual', whereby a residual is divided by the standard deviation of the dependent variable. The Pearson residual can be used when the variability of $\hat{\beta}$ is disregarded in the underlying assumptions.
\newpage

%--------------------------------------------------%
\subsection{Residual Analysis for Linear Models, LME models and GLMs}

\textbf{Keywords:}

\begin{itemize}
	\item Residuals (\emph{Beginners}), 
	\item Testing the Assumption of Normality (\emph{Beginners})
	\item Diagnostic Plots with the \texttt{plot} function
	\item Cook's Distance
	\item DFFits and DFBeta
	\item Standardized and Studentized Residuals
	\item Influence Leverage and Outlierness
\end{itemize}

%--------------------------------------------------%
%---------------------------------------------------------------------------%
\newpage
\section{Residual diagnostics} %1.3
For classical linear models, residual diagnostics are typically implemented as a plot of the observed residuals and the predicted values. A visual inspection for the presence of trends inform the analyst on the validity of distributional assumptions, and to detect outliers and influential observations.



%--Marginal and Conditional Residuals

\subsection{Residuals diagnostics in mixed models}

%schabenberger
The marginal and conditional means in the linear mixed model are
$E[\boldsymbol{Y}] = \boldsymbol{X}\boldsymbol{\beta}$ and
$E[\boldsymbol{Y|\boldsymbol{u}}] = \boldsymbol{X}\boldsymbol{\beta} + \boldsymbol{Z}\boldsymbol{u}$, respectively.

A residual is the difference between an observed quantity and its estimated or predicted value. In the mixed
model you can distinguish marginal residuals $r_m$ and conditional residuals $r_c$. 


\subsection{Marginal and Conditional Residuals}

A marginal residual is the difference between the observed data and the estimated (marginal) mean, $r_{mi} = y_i - x_0^{\prime} \hat{b}$
A conditional residual is the difference between the observed data and the predicted value of the observation,
$r_{ci} = y_i - x_i^{\prime} \hat{b} - z_i^{\prime} \hat{\gamma}$

In linear mixed effects models, diagnostic techniques may consider `conditional' residuals. A conditional residual is the difference between an observed value $y_{i}$ and the conditional predicted value $\hat{y}_{i} $.

\[ \hat{epsilon}_{i} = y_{i} - \hat{y}_{i} = y_{i} - ( X_{i}\hat{beta} + Z_{i}\hat{b}_{i}) \]

However, using conditional residuals for diagnostics presents difficulties, as they tend to be correlated and their variances may be different for different subgroups, which can lead to erroneous conclusions.

%1.5
%http://support.sas.com/documentation/cdl/en/statug/63033/HTML/default/viewer.htm#statug_mixed_sect024.htm






\begin{equation}
r_{mi}=x^{T}_{i}\hat{\beta}
\end{equation}

\subsection{Marginal Residuals}
\begin{eqnarray}
\hat{\beta} &=& (X^{T}R^{-1}X)^{-1}X^{T}R^{-1}Y \nonumber \\
&=& BY \nonumber
\end{eqnarray}

%---------------------------------------------------------------------------%
\newpage
\section{Standardized and studentized residuals} %1.4
%--Studentized and Standardized Residuals

To alleviate the problem caused by inconstant variance, the residuals are scaled (i.e. divided) by their standard deviations. This results in a \index{standardized residual}`standardized residual'. Because true standard deviations are frequently unknown, one can instead divide a residual by the estimated standard deviation to obtain the \index{studentized residual}`studentized residual. 

\subsection{Standardization} %1.4.1

A random variable is said to be standardized if the difference from its mean is scaled by its standard deviation. The residuals above have mean zero but their variance is unknown, it depends on the true values of $\theta$. Standardization is thus not possible in practice.

\subsection{Studentization} %1.4.2
Instead, you can compute studentized residuals by dividing a residual by an estimate of its standard deviation. 

\subsection{Internal and External Studentization} %1.4.3
If that estimate is independent of the $i-$th observation, the process is termed \index{external studentization}`external studentization'. This is usually accomplished by excluding the $i-$th observation when computing the estimate of its standard error. If the observation contributes to the
standard error computation, the residual is said to be \index{internally studentization}internally studentized.

Externally \index{studentized residual} studentized residual require iterative influence analysis or a profiled residuals variance.


\subsection{Computation}%1.4.4

The computation of internally studentized residuals relies on the diagonal entries of $\boldsymbol{V} (\hat{\theta})$ - $\boldsymbol{Q} (\hat{\theta})$, where $\boldsymbol{Q} (\hat{\theta})$ is computed as

\[ \boldsymbol{Q} (\hat{\theta}) = \boldsymbol{X} ( \boldsymbol{X}^{\prime}\boldsymbol{Q} (\hat{\theta})^{-1}\boldsymbol{X})\boldsymbol{X}^{-1} \]

\subsection{Pearson Residual}%1.4.5

Another possible scaled residual is the \index{Pearson residual} `Pearson residual', whereby a residual is divided by the standard deviation of the dependent variable. The Pearson residual can be used when the variability of $\hat{\beta}$ is disregarded in the underlying assumptions.

%---------------------------------------------------------------------------%

\newpage
%--------------------------------------------------------------------Diagnostics%
\section{Diagnostics}

%http://www.artifex.org/~meiercl/R_statistics_guide.pdf
\subsection{Identifying outliers with a LME model object}

The process is slightly different than with standard LME model objects, since the \textbf{\emph{influence}}
function does not work on lme model objects. Given \textbf{\emph{mod.lme}}, we can use the plot function to
identify outliers.
%----------------------%
\subsection{Diagnostics for Random Effects}
Empirical best linear unbiased predictors EBLUPS provide the a useful way of diagnosing random effects.

EBLUPs are also known as ``shrinkage estimators" because they tend to be smaller than the estimated effects would be if they were computed by treating a random factor as if it was fixed (West etal )




\newpage
%---------------------------------------------------------------------------%
\section{Introduction}%1.1
In classical linear models model diagnostics have been become a required part of any statistical analysis, and the methods are commonly available in statistical packages and standard textbooks on applied regression. However it has been noted by several papers that model diagnostics do not often accompany LME model analyses.
Model diagnostic techniques determine whether or not the distributional assumptions are satisfied, and to assess the influence of unusual observations.

\subsection{Model Data Agreement} %1.1.1
\citet{schabenberger} describes the examination of model-data agreement as comprising several elements; residual analysis, goodness of fit, collinearity diagnostics and influence analysis.

\subsection{Influence Diagnostics: Basic Idea and Statistics} %1.1.2
%http://support.sas.com/documentation/cdl/en/statug/63033/HTML/default/viewer.htm#statug_mixed_sect024.htm

The general idea of quantifying the influence of one or more observations relies on computing parameter estimates based on all data points, removing the cases in question from the data, refitting the model, and computing statistics based on the change between full-data and reduced-data estimation. 



\subsection{Influence Analysis for LME Models} %1.1.3
The linear mixed effects model is a useful methodology for fitting a wide range of models. However, linear mixed effects models are known to be sensitive to outliers. \citet{CPJ} advises that identification of outliers is necessary before conclusions may be drawn from the fitted model.

Standard statistical packages concentrate on calculating and testing parameter estimates without considering the diagnostics of the model.The assessment of the effects of perturbations in data, on the outcome of the analysis, is known as statistical influence analysis. Influence analysis examines the robustness of the model. Influence analysis methodologies have been used extensively in classical linear models, and provided the basis for methodologies for use with LME models.
Computationally inexpensive diagnostics tools have been developed to examine the issue of influence \citep{Zewotir}.
Studentized residuals, error contrast matrices and the inverse of the response variance covariance matrix are regular components of these tools.

\subsection{Influence Statistics for LME models} %1.1.4
Influence statistics can be coarsely grouped by the aspect of estimation that is their primary target:
\begin{itemize}
	\item overall measures compare changes in objective functions: (restricted) likelihood distance (Cook and Weisberg 1982, Ch. 5.2)
	\item influence on parameter estimates: Cook's  (Cook 1977, 1979), MDFFITS (Belsley, Kuh, and Welsch 1980, p. 32)
	\item influence on precision of estimates: CovRatio and CovTrace
	\item influence on fitted and predicted values: PRESS residual, PRESS statistic (Allen 1974), DFFITS (Belsley, Kuh, and Welsch 1980, p. 15)
	\item outlier properties: internally and externally studentized residuals, leverage
\end{itemize}
%---------------------------------------------------------------------------%


\subsection{What is Influence} %1.1.5

Broadly defined, influence is understood as the ability of a single or multiple data points, through their presence or absence in the data, to alter important aspects of the analysis, yield qualitatively different inferences, or violate assumptions of the statistical model. The goal of influence analysis is not primarily to mark data
points for deletion so that a better model fit can be achieved for the reduced data, although this might be a result of influence analysis \citep{schabenberger}.

%-------%
\subsection{Quantifying Influence}  %1.1.6

The basic procedure for quantifying influence is simple as follows:

\begin{itemize}
	\item Fit the model to the data and obtain estimates of all parameters.
	\item Remove one or more data points from the analysis and compute updated estimates of model parameters.
	\item Based on full- and reduced-data estimates, contrast quantities of interest to determine how the absence of the observations changes the analysis.
\end{itemize}

\citet{cook86} introduces powerful tools for local-influence assessment and examining perturbations in the assumptions of a model. In particular the effect of local perturbations of parameters or observations are examined.


%---------------------------------------------------------------------------%
\newpage
\section{Extension of techniques to LME Models} %1.2

Model diagnostic techniques, well established for classical models, have since been adapted for use with linear mixed effects models.Diagnostic techniques for LME models are inevitably more difficult to implement, due to the increased complexity.

Beckman, Nachtsheim and Cook (1987) \citet{Beckman} applied the \index{local influence}local influence method of Cook (1986) to the analysis of the linear mixed model.

While the concept of influence analysis is straightforward, implementation in mixed models is more complex. Update formulae for fixed effects models are available only when the covariance parameters are assumed to be known.

If the global measure suggests that the points in $U$ are influential, the nature of that influence should be determined. In particular, the points in $U$ can affect the following

\begin{itemize}
	\item the estimates of fixed effects,
	\item the estimates of the precision of the fixed effects,
	\item the estimates of the covariance parameters,
	\item the estimates of the precision of the covariance parameters,
	\item fitted and predicted values.
\end{itemize}


%---------------------------------------------------------------------------%
\newpage


\subsection{Case Deletion Diagnostics for Mixed Models}

\citet{Christiansen} notes the case deletion diagnostics techniques have not been applied to linear mixed effects models and seeks to develop methodologies in that respect.

\citet{Christiansen} develops these techniques in the context of REML

\subsection{Methods and Measures}
The key to making deletion diagnostics useable is the development of efficient computational formulas, allowing one to obtain the \index{case deletion diagnostics} case deletion diagnostics by making use of basic building blocks, computed only once for the full model.

\citet{Zewotir} lists several established methods of analyzing influence in LME models. These methods include \begin{itemize}
	\item Cook's distance for LME models,
	\item \index{likelihood distance} likelihood distance,
	\item the variance (information) ration,
	\item the \index{Cook-Weisberg statistic} Cook-Weisberg statistic,
	\item the \index{Andrews-Prebigon statistic} Andrews-Prebigon statistic.
\end{itemize}


%---------------------------------------------------------------------------%
\newpage
\section{Influence analysis} %1.7

Likelihood based estimation methods, such as ML and REML, are sensitive to unusual observations. Influence diagnostics are formal techniques that assess the influence of observations on parameter estimates for $\beta$ and $\theta$. A common technique is to refit the model with an observation or group of observations omitted.

\citet{west} examines a group of methods that examine various aspects of influence diagnostics for LME models.
For overall influence, the most common approaches are the `likelihood distance' and the `restricted likelihood distance'.

\subsection{Cook's 1986 paper on Local Influence}%1.7.1
Cook 1986 introduced methods for local influence assessment. These methods provide a powerful tool for examining perturbations in the assumption of a model, particularly the effects of local perturbations of parameters of observations.

The local-influence approach to influence assessment is quitedifferent from the case deletion approach, comparisons are of
interest.



\subsection{Overall Influence}
An overall influence statistic measures the change in the objective function being minimized. For example, in
OLS regression, the residual sums of squares serves that purpose. In linear mixed models fit by
\index{maximum likelihood} maximum likelihood (ML) or \index{restricted maximum likelihood} restricted maximum likelihood (REML), an overall influence measure is the \index{likelihood distance} likelihood distance [Cook and Weisberg ].

\end{document}
