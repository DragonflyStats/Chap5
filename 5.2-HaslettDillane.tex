\documentclass[Main.tex]{subfiles}
\begin{document}
	
	\section{Case Deletion Diagnostics for LME models}
	
	%%% Haslett \& Dillane (19XX) }
	
	Haslett \& Dillane (19XX) remark that linear mixed effects models
didn't experience a corresponding growth in the use of deletion
diagnostics, adding that \citet{McCullSearle} makes no mention of
diagnostics whatsoever.

%%%\citet{christensen}

Christensen (19XX)  describes three propositions that are required
for efficient case-deletion in LME models. The first proposition
decribes how to efficiently update $V$ when the $i$th element is
deleted.
\begin{equation}
V_{[i]}^{-1} = \Lambda_{[i]} - \frac{\lambda
	\lambda\prime}{\nu^{}ii}
\end{equation}


The second of Christensen's propostions is the following set of
equations, which are variants of the Sherman Wood bury updating
formula.
\begin{eqnarray}
X'_{[i]}V_{[i]}^{-1}X_{[i]} &=& X' V^{-1}X -
\frac{\hat{x}_{i}\hat{x}'_{i}}{s_{i}}\\
(X'_{[i]}V_{[i]}^{-1}X_{[i]})^{-1} &=& (X' V^{-1}X)^{-1} +
\frac{(X' V^{-1}X)^{-1}\hat{x}_{i}\hat{x}' _{i}
	(X' V^{-1}X)^{-1}}{s_{i}- \bar{h}_{i}}\\
X'_{[i]}V_{[i]}^{-1}Y_{[i]} &=& X\prime V^{-1}Y -
\frac{\hat{x}_{i}\hat{y}' _{i}}{s_{i}}
\end{eqnarray}








In LME models, fitted by either ML or REML, an important overall
influence measure is the likelihood distance \citep{cook82}. The
procedure requires the calculation of the full data estimates
$\hat{\psi}$ and estimates based on the reduced data set
$\hat{\psi}_{(U)}$. The likelihood distance is given by
determining


\begin{eqnarray}
LD_{(U)} &=& 2\{l(\hat{\psi}) - l( \hat{\psi}_{(U)}) \}\\
RLD_{(U)} &=& 2\{l_{R}(\hat{\psi}) - l_{R}(\hat{\psi}_{(U)})\}
\end{eqnarray}


% Haslett Dillane
%==================================================================%
Haslett \& Dillane (199X) offers an
procedure to assess the influences for the variance components
within the linear model, complementing the existing methods for
the fixed components. 


The essential problem is that there is no
useful updating procedures for $\hat{V}$, or for $\hat{V}^{-1}$.
Haslett \& Dillane (199X) propose an alternative , and
computationally inexpensive approach, making use of the
`\texttt{delete=replace}' identity.

\citet{Haslett99} considers the effect of `leave k out'
calculations on the parameters $\beta$ and $\sigma^{2}$, using
several key results from \citet{HaslettHayes} on partioned
matrices.

% - Haslett \& Dillane (199X)  - Haslett \& Dillane (19XX) }
% - Haslett (1999) - \citet{Haslett99}


%---------------------------------------------------------------------------%
\newpage
\section{Haslett's Analysis} %2.5
For fixed effect linear models with correlated error structure Haslett (1999) showed that the effects on
the fixed effects estimate of deleting each observation in turn could be cheaply computed from the fixed effects model predicted residuals.

\bibliography{DB-txfrbib}
\end{document}
