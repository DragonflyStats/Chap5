\documentclass[Chap5amain.tex]{subfiles}
\begin{document}
\section{Conditional and Marginal Residuals}
Conditional residuals include contributions from both fixed and random effects, whereas marginal residuals include contribution from only fixed effects.

Suppose the linear mixed-effects model lmehas an n-by-p fixed-effects design matrix X and an n-by-q random-effects design matrix Z. 

%Also, suppose the p-by-1 estimated fixed-effects vector is $\hat{\beta}}$ , and the q-by-1 estimated best linear unbiased predictor (BLUP) 
%vector of random effects is $\hat{b}}$ . The fitted conditional response is

\[ \hat{y}_{Cond} = X \hat{\beta} + Z \hat{b} \]

and the fitted marginal response is


\[ \hat{y}_{Mar} = X \hat{\beta} \]

residuals can return three types of residuals: raw, Pearson, and standardized. For any type, you can compute the conditional or the marginal residuals. For example, the conditional raw residual is


\[ r_{Cond} = y - X \hat{\beta} - Z \hat{b} \]

and the marginal raw residual is



\[ r_{Mar} = y - X \hat{\beta} \]

%------------------------------------------------------------------%
\newpage

Marginal residuals:

\[y - X\beta = Z \eta +\epsilon \]
\begin{itemize}
\item
Should be mean 0, but may show grouping structure
\item
May not be homoskedastic!
\item
Good for checking fixed effects, just like linear regr.
\end{itemize}
%----------------------------------------------------%
Conditional residuals:
\[y - X\beta - Z \eta = \epsilon \]
\begin{itemize}
\item
Should be mean zero with no grouping structure
\item
Should be homoskedastic!
\item
Good for checking normality of outliers
\end{itemize}

%-----------------------------------------------------%
Random effects:
\[y - X\beta -\epsilon = Z \eta \]
\begin{itemize}
\item
Should be mean zero with no grouping structure
\item
May not be be homoskedastic!
\end{itemize}

\end{document}
