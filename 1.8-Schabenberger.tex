\documentclass[Main.tex]{subfiles}
\begin{document}
	
	% 1.  Cook's Distance 1977
	% 2.  Extension to 
	% 3.  Demidenko 
	% 4.  Schabenburger
	%---------------------------------------------------------------% 
	
	\textit{schabenberger} examines the use and implementation of
	influence measures in LME models.
	
	Influence is understood to be the ability of a single or multiple
	data points, through their presences or absence in the data, to
	alter important aspects of the analysis, yield qualitatively
	different inferences, or violate assumptions of the statistical
	model (\textit{schabenberger}).
	
	Outliers are the most noteworthy data points in an analysis, and
	an objective of influence analysis is how influential they are,
	and the manner in which they are influential.
	
	\textit{schabenberger} describes a simple procedure for quantifying
	influence. Firstly a model should be fitted to the data, and
	estimates of the parameters should be obtained. The second step is
	that either single of multiple data points, specifically outliers,
	should be omitted from the analysis, with the original parameter
	estimates being updated. This is known as `leave one out \ leave k
	out' analysis. The final step of the procedure is comparing the
	sets of estimates computed from the entire and reduced data sets
	to determine whether the absence of observations changed the
	analysis.
	
	
	
	A residual is the difference between an observed quantity and its
	estimated or predicted value. In LME models, there are two types
	of residuals, marginal residuals and conditional residuals. A
	marginal residual is the difference between the observed data and
	the estimated marginal mean. A conditional residual is the
	difference between the observed data and the predicted value of
	the observation. In a model without random effects, both sets of
	residuals coincide.
	
	\textit{schabenberger} notes that it is not always possible to
	derive influence statistics necessary for comparing full- and
	reduced-data parameter estimates. 
	
	
	\begin{abstract}
		\noindent This paper reviews the use of diagnostic measures for LME models in SAS. This text has been widely cited by texts that don't deal with SAS implementations.
	\end{abstract}
	

\newpage

	
	\section{Introduction (Page 1)}
	
	Linear models for uncorrelated data have well established measures to gauge the influence of one or more
	observations on the analysis. For such models, closed-form update expressions allow efficient computations
	without refitting the model. When similar notions of statistical influence are applied to mixed models,
	things are more complicated. Removing data points affects fixed effects and covariance parameter estimates.
	Update formulas for “leave-one-out” estimates typically fail to account for changes in covariance
	parameters. Moreover, in repeated measures or longitudinal studies, one is often interested in multivariate
	influence, rather than the impact of isolated points. This paper examines extensions of influence measures
	in linear mixed models and their implementation in the MIXED procedure.
	INTRODUCTION
	A statistical model, whether of the fixed-effects or mixed-effects variety, represents how you think your data
	were generated. Following model specification and estimation, it is of interest to explore the model-data
	agreement by raising questions such as
	\begin{itemize}
		\item Does the model-data agreement support the model assumptions?
		\item Should model components be refined, and if so, which components? For example, should regressors
		be added or removed, and is the covariation of the observations modeled properly?
		\item Are the results sensitive to model and/or data? Are individual data points or groups of cases particularly
		influential on the analysis?
	\end{itemize}

	
	%=========================================================================%
	\subsection*{Model Validation Framework}
	In classical linear models, this examination of model-data agreement has traditionally revolved around
	\begin{itemize}
	\item the informal, graphical examination of estimates of model errors to assess the quality of distributional
	assumptions: residual analysis
	
	\item  overall measures of goodness-of-fit
	\item the quantitative assessment of the inter-relationship of model components; for example, collinearity
	diagnostics
	\item the qualitative and quantitative assessment of influence of cases on the analysis: influence analysis.
\end{itemize}
	The sensitivity of a model is studied through measures that express its stability under perturbations. You
	are not interested in a model that is either overly stable or overly sensitive. Changes in the data or model
	components should produce commensurate changes in the model output. The difficulty is to determine
	when the changes are substantive enough to warrant further investigation, possibly leading to a reformulation
	of the model or changes in the data (such as dropping outliers). This paper is primarily concerned
	with stability of linear mixed models to perturbations of the data; that is, with influence analysis. Broadly
	defined, “influence” is understood as the ability of a single or multiple data points, through their presence
	or absence in the data, to alter important aspects of the analysis, yield qualitatively different inferences, or
	violate assumptions of the statistical model. The goal of influence analysis is not primarily to mark data
	points for deletion so that a better model fit can be achieved for the reduced data, although this might be a
	result of influence analysis. The goal is rather to determine which cases are influential and the manner in
	which they are important to the analysis. Outliers, for example, may be the most noteworthy data points in
	an analysis. They can point to a model breakdown and lead to development of a better model.
	
	
	
	
	In recent years, mixed models have become invaluable tools in the analysis of experimental and observational
	data. In these models, more than one term can be subject to random variation. Mixed model
	technology enables you to analyze complex experimental data with hierarchical random processes, temporal,
	longitudinal, and spatial data, to name just a few important applications. 
	
	\subsection*{Stating the LME Model}
	The general linear mixed
	model is
	\[
	Y = X\beta + Zu + \varepsilon\]
	where Y is a $(n\times1)$ vector of observed data, X is an $(n\times p)$ fixed-effects design or regressor matrix of rank
	k, Z is a $(n \times g)$ random-effects design or regressor matrix, $u$ is a $(g \times 1)$ vector of random effects, and $\varepsilon$ is
	an $(n\times1)$ vector of model errors (also random effects). The distributional assumptions made by the MIXED
	procedure are as follows: γ is normal with mean 0 and variance G; $\varepsilon$ is normal with mean 0 and variance
	R; the random components $u$ and $\varepsilon$ are independent. Parameters of this model are the fixed-effects β and
	all unknowns in the variance matrices G and R. The unknown variance elements are referred to as the
	covariance parameters and collected in the vector $theta$.
	%===========================================================================%
	The concept of critiquing the model-data agreement applies in mixed models in the same way as in linear
	fixed-effects models. In fact, because of the more complex model structure, you can argue that model and
	data diagnostics are even more important. For example, you are not only concerned with capturing the
	important variables in the model. You are also concerned with “distributing” them correctly between the
	fixed and random components of the model. The mixed model structure presents unique and interesting
	challenges that prompt us to reexamine the traditional ideas of influence and residual analysis.
	%==========================================================================%
	This paper presents the extension of traditional tools and statistical measures for influence and residual
	analysis to the linear mixed model and demonstrates their implementation in the MIXED procedure (experimental
	features in SAS 9.1). The remainder of this paper is organized as follows. The “Background” section
	briefly discusses some mixed model estimation theory and the challenges to model diagnosis that result
	from it.
	
%	 The diagnostics implemented in the MIXED procedure are discussed in the “Residual Diagnostics
%	in the MIXED Procedure” section (page 3) and the “Influence Diagnostics in the MIXED Procedure” section
%	(page 5). The syntax options and suboptions you use to request the various diagnostics are briefly sketched
%	in the “Syntax” section (page 9). The presentation concludes with an example.
%	
%	
	%====================================================================================================================%
	\newpage
	\subsection*{INFLUENCE DIAGNOSTICS IN THE MIXED PROCEDURE}
	Key to the implementations of influence diagnostics in the MIXED procedure is the attempt to quantify
	influence, where possible, by drawing on the basic definitions of the various statistics in the classical linear
	model. On occasion, quantification is not possible. Assume, for example, that a data point is removed
	and the new estimate of the G matrix is not positive definite. This may occur if a variance component
	estimate now falls on the boundary of the parameter space. Thus, it may not be possible to compute certain
	influence statistics comparing the full-data and reduced-data parameter estimates. However, knowing that
	a new singularity was encountered is important qualitative information about the data point’s influence on
	the analysis.
	
	The basic procedure for quantifying influence is simple:
	
	\begin{enumerate}
		\item Fit the model to the data and obtain estimates of all parameters.
		\item Remove one or more data points from the analysis and compute updated estimates of model parameters.
		\item Based on full- and reduced-data estimates, contrast quantities of interest to determine how the absence
		of the observations changes the analysis.
	\end{enumerate}
	We use the subscript (U) to denote quantities obtained without the observations in the set U. For example,
	βb
	(U) denotes the fixed-effects “\textit{\textbf{leave-U-out}}” estimates. Note that the set U can contain multiple observations.
	
	
	%===================================================================================
	If the global measure suggests that the points in U are influential, you should next determine the nature of
	that influence. In particular, the points can affect
	\begin{itemize}
		\item the estimates of fixed effects
		\item the estimates of the precision of the fixed effects
		\item the estimates of the covariance parameters
		\item the estimates of the precision of the covariance parameters
		\item fitted and predicted values
	\end{itemize}
	
	It is important to further decompose the initial finding to determine whether data points are actually troublesome.
	Simply because they are influential “somehow”, should not trigger their removal from the analysis or
	a change in the model. For example, if points primarily affect the precision of the covariance parameters
	without exerting much influence on the fixed effects, then their presence in the data may not distort hypothesis
	tests or confidence intervals about β. They will only do so if your inference depends on an estimate of the
	precision of the covariance parameter estimates, as is the case for the Satterthwaite and Kenward-Roger
	degrees of freedom methods and the standard error adjustment associated with the DDFM=KR option.
	
	
\newpage




%---------------------------------------------------------------------------%
\newpage
\section{Iterative and non-iterative influence analysis} %1.13
\citet{schabenberger} highlights some of the issue regarding implementing mixed model diagnostics.

A measure of total influence requires updates of all model parameters.

however, this doesnt increase the procedures execution time by the same degree.
\subsection{Iterative Influence Analysis}

%----schabenberger page 8
For linear models, the implementation of influence analysis is straightforward.
However, for LME models, the process is more complex. Update formulas for the fixed effects are available only when the covariance parameters are assumed to be known. A measure of total influence requires updates of all model parameters.
This can only be achieved in general is by omitting observations, then refitting the model.

\citet{schabenberger} describes the choice between \index{iterative influence analysis} iterative influence analysis and \index{non-iterative influence analysis} non-iterative influence analysis.


%------------------------------------------------------------%
\subsection{Summary of Paper}
%Summary of Schabenberger
Standard residual and influence diagnostics for linear models can be extended to LME models.
The dependence of the fixed effects solutions on the covariance parameters has important ramifications on the perturbation analysis.	
Calculating the studentized residuals-And influence statistics whereas each software procedure can calculate both conditional and marginal raw residuals, only SAs Proc Mixed is currently the only program that provide studentized residuals Which ave preferred for model diagnostics. The conditional Raw residuals ave not well suited to detecting outliers as are the studentized conditional residuals. (schabenbege r)


LME are flexible tools for the analysis of clustered and repeated measurement data. LME extend the capabilities of standard linear models by allowing unbalanced and missing data, as long as the missing data are MAR. Structured covariance matrices for both the random effects G and the residuals R. missing at Random.

A conditional residual is the difference between the observed valve and the predicted valve of a dependent variable- Influence diagnostics are formal techniques that allow the identification observation that heavily influence estimates of parameters.
To alleviate the problems with the interpretation of conditional residuals that may have unequal variances, we consider sealing.
Residuals obtained in this manner ave called studentized residuals.



\subsection{ITERATIVE VS. NONITERATIVE INFLUENCE ANALYSIS}
While the basic idea of influence analysis is straightforward, the implementation in mixed models can be
tricky. For example, update formulas for the fixed effects are available only when the covariance parameters
are assumed to be known. At most the profiled residual variance can be updated without refitting the model.
A measure of total influence requires updates of all model parameters, and the only way that this can be
achieved in general is by removing the observations in question and refitting the model. Because this “bruteforce”
method involves iterative reestimation of the covariance parameters, it is termed iterative influence
analysis. Reliance on closed-form update formulas for the fixed effects without updating the (un-profiled)
covariance parameters is termed a noniterative influence analysis.
An iterative analysis seems like a costly, computationally intensive enterprise. If you compute iterative
influence diagnostics for all n observations, then a total of n + 1 mixed models are fit iteratively. This does
not imply, of course, that the procedure’s execution time increases n-fold. Keep in mind that
\begin{itemize}
	\item iterative reestimation always starts at the converged full-data estimates. If a data point is not influential,
	then its removal will have little effect on the objective function and parameter estimates. Within
	one or two iterations, the process should arrive at the reduced-data estimates.
	\item if complete reestimation does require many iterations, then this is important information in itself. The
	likelihood surface has probably changed drastically, and the reduced-data estimates are moving away
\end{itemize}
from the full-data estimates.

\newpage
\subsection*{SUMMARY AND CONCLUSIONS}
Standard residual and influence diagnostics for linear models can be extended to linear mixed models. The
dependence of fixed-effects solutions on the covariance parameter estimates has important ramifications
in perturbation analysis. To gauge the full impact of a set of observations on the analysis, covariance
parameters need to be updated, which requires refitting of the model. 

The experimental INFLUENCE
option of the MODEL statement in the MIXED procedure (SAS 9.1) enables you to perform iterative and
noniterative influence analysis for individual observations and sets of observations.
%------------------------------------------------------------%
The conditional (subject-specific) and marginal (population-averaged) formulations in the linear mixed model
enable you to consider conditional residuals that use the estimated BLUPs of the random effects, and
marginal residuals which are deviations from the overall mean. Residuals using the BLUPs are useful to
diagnose whether the random effects components in the model are specified correctly, marginal residuals
are useful to diagnose the fixed-effects components. Both types of residuals are available in SAS 9.1 as an
experimental option of the MODEL statement in the MIXED procedure.
%------------------------------------------------------------%
It is important to note that influence analyses are performed under the assumption that the chosen model
is correct. Changing the model structure can alter the conclusions. Many other variance models have been
fit to the data presented in the repeated measures example. You need to see the conclusions about which
model component is affected in light of the model being fit. For example, modeling these data with a random
intercept and random slope for each child or an unstructured covariance matrix will affect your conclusions
about which children are influential on the analysis and how this influence manifests itself.


\newpage

	
	%---------------------------------------------------------------%
	\section*{Schabenberger: Summary and Conclusions}
	\begin{itemize}
		\item Standard residual and influence diagnostics for linear models can be extended to linear mixed models. The dependence of fixed-effects solutions on the covariance parameter estimates has important ramifications in perturbation analysis. 
		\item To gauge the full impact of a set of observations on the analysis, covariance parameters need to be updated, which requires refitting of the model. 
		\item The experimental INFLUENCE option of the MODEL statement in the MIXED procedure (SAS 9.1) enables you to perform iterative and noniterative influence analysis for individual observations and sets of observations.
		
		\item The conditional (subject-specific) and marginal (population-averaged) formulations in the linear mixed model enable you to consider conditional residuals that use the estimated BLUPs of the random effects, and marginal residuals which are deviations from the overall mean. 
		\item Residuals using the BLUPs are useful to diagnose whether the random effects components in the model are specified correctly, marginal residuals are useful to diagnose the fixed-effects components. 
		\item Both types of residuals are available in SAS 9.1 as an experimental option of the MODEL statement in the MIXED procedure.
		
		\item It is important to note that influence analyses are performed under the assumption that the chosen model is correct. Changing the model structure can alter the conclusions. Many other variance models have been fit to the data presented in the repeated measures example. You need to see the conclusions about which model component is affected in light of the model being fit.
		\item  For example, modeling these data with a random intercept and random slope for each child or an unstructured covariance matrix will affect your conclusions about which children are influential on the analysis and how this influence manifests itself.
	\end{itemize}

\end{document}
