%- The Mahalanobis distance is a measure of the distance between a point P and a distribution D, introduced by P. C. Mahalanobis in 1936.
4 It is a multi-dimensional generalization of the idea of measuring how many standard deviations away P is from the mean of D. 

This distance is zero if P is at the mean of D, and grows as P moves away from the mean: Along each principal component axis, it measures the 
number of standard deviations from P to the mean of D. If each of these axes is rescaled to have unit variance, then Mahalanobis distance corresponds to standard Euclidean distance in the transformed space. Mahalanobis distance is thus unitless and scale-invariant, and takes into account the correlations of the data set.

%============================================================================%
\subsection*{Description
}
\texttt{influence()} is the workhorse function of the \texttt{influence.ME} package. 


Based on a priorly estimated mixed effects regression model (estimated using lme4), the \texttt{influence()} function iteratively 

modifies the mixed effects model to neutralize the effect a grouped set of data has on the parameters, and which 

returns returns the fixed parameters of these iteratively modified models. 

These are used to compute measures of influential data.




\subsection*{Usage
}
\begin{framed}
\begin{verbatim}

influence(model, group=NULL, select=NULL, obs=FALSE, 
    gf="single", count = FALSE, delete=TRUE, ...)
\
\end{verbatim}
\end{framed}


The \texttt{influence()} function was known as the \texttt{estex()} command in previous versions of the influence.ME pacakge
%===========================================================================%
%- http://support.sas.com/documentation/cdl/en/statug/63347/HTML/default/statug_reg_sect040.htm
\texttt{dfbeta()}


The DFBETAS statistics are the scaled measures of the change in each parameter estimate and are calculated by deleting the th observation:
 		 	 
where  is the th element of .
In general, large values of DFBETAS indicate observations that are influential in estimating a given parameter. Belsley, Kuh, and Welsch (1980) recommend 2 as a general cutoff value to indicate influential observations and  as a size-adjusted cutoff.

%---------------------------------------------------------------------------% 
Probably the most popular 
tools is DFBETA. DFBETA is a measure found for each observation in a dataset. The DFBETA for a 
particular observation is the difference between the regression coefficient for an included variable (say 
age, or education in our well-worn salary example) calculated for all of the data and the regression 
coefficient calculated with the observation deleted, scaled by the standard error calculated with the 
observation deleted. The cut-off value for DFBETAs is 2/sqrt(n), where n is the number of observations. 
However, another cut-off is to look for observations with a value greater than 1.00. Here cutoff means, 
“this observation could be overly influential on the estimated coefficient.” 
%==========================================================================%
%WIKIPEDIA
DFFITS is a diagnostic meant to show how influential a point is in a statistical regression. It was proposed in 1980.[1] It is defined as the change ("DFFIT"), in the predicted value for a point, obtained when that point is left out of the regression, "Studentized" by dividing by the estimated standard deviation of the fit at that point:
\[ \text{DFFITS} = {\widehat{y_i} - \widehat{y_{i(i)}} \over s_{(i)} \sqrt{h_{ii}}}\]
