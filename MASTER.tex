\documentclass{article}
\usepackage[utf8]{inputenc}
\usepackage{framed}

\title{Influence Techniques for  LME models}
\author{kobriendublin }
\date{February 2015}


\begin{document}
\author{Kevin O'Brien}
\title{Implementation of Model Diagnostics for LME models with \texttt{R}}

\tableofcontents
%====================================================================%
\newpage

\section{Introduction to Influence Analysis}

Outliers and detection of influential observations is an important step in the analysis of a data set. There are several ways of evaluating the influence of perturbations in the data set and in the model given the parameter estimates. 

The basic rationale behind measuring influential cases is that when iteratively single units are omitted
from the data, models based on these data should not produce substantially different estimates.
%===================================================================== %
\subsection{Well Known Influence Measures}

\textit{``Regression Diagnostics: Identifying Influential Data
and Source of Collinearity (1980)"} by Belsley,Kuh,\& Welsch is a landmark text in the field of residual diagnostics, and
provides a foundation for much of the subsequent work.

\begin{description}
\item[Cook's Distance] Cook’s Distance is a measure indicating to what extent model parameters are influenced by (a set
of) influential data on which the model is based.
\item[DFBETAS] DFBETAS (standardized difference of the beta) is a measure that standardizes the absolute difference
in parameter estimates between a (mixed effects) regression model based on a full set of
data, and a model from which a (potentially influential) subset of data is removed. A value for
DFBETAS is calculated for each parameter in the model separately.
\end{description}

\subsection{Overview of R implementations}
Further to previous material, an appraisal of the current state of development (or lack thereof) for current implemenations for LME models, particularly for \texttt{nlme} and \texttt{lme4} fitted models.

Crucially, a review of internet resources indicates that almost all of the progress in this regard has been done for \texttt{lme4} fitted models, specifically the \textit{Influence.ME} \texttt{R} package. (Nieuwenhuis et al 2014)
Conversely there is very little for \texttt{nlme} models. One would immediately look at the current development workflow for both packages.
 
%======================%
% Douglas Bates

As an aside, Douglas Bates was arguably the most prominent \texttt{R} developer working in the LME area. 
However Bates has now prioritised the development of LME models in another computing environment , i.e Julia. 
% The current version of this is XXXX

%======================%
% nlme

With regards to \texttt{nlme}, the package is now maintained by the \texttt{R} core development team. The most recent major text is by Galecki \& Burzykowski, who have published \textit{ Linear Mixed Effects Models using \texttt{R}. }
Also, the accompanying \texttt{R} package, nlmeU package is under current development, with a version being released $0.70-3$.


%======================%
% lme4 and influence.ME

The \textbf{lme4} pacakge is used to fit linear and generalized linear mixed-effects models in the R environment.
The \textbf{lme4} package is also under active development, under the leadership of Ben Bolker (McMaster Uni., Canada).


%=====================%
\subsection{Important Consideration for MCS}

The key issue is that \texttt{nlme} allows for the particular specification of Roy's Model, speciifically direct specification of the VC matrices for within subject and between subject residuals.
The \texttt{lme4} package does not allow for Roy's Model, for reasons that will identified shortly.
To advance the ideas that eminate from Roys' paper, one is required to use the \texttt{nlme} context. However, to take advantage of the infrastructure already provided for \texttt{lme4} models, one may change the research question away from that of Roy's paper. 
To this end, an exploration of what textbf{influence.ME} can accomplished is merited.


%==========================================================================%
\newpage


%============================================================%
\newpage
\section{Diagnostic Tools for the nlme package}


With the nlme package, the generic function \texttt{lme()} fits a linear mixed-effects model in the formulation described in Laird and Ware (1982) but allowing for nested random effects. 

The within-group errors are allowed to be correlated and/or have unequal variances, which is very important in fitting the models for Roy's Tests

The nlme package has a limited set of diagnostic tools that can be used to assess the model fit. A review of the package manual is sufficient to get a sense of the package's capability in that regard.

\begin{description}
\item[residuals.lme] 
The residuals at level $i$ are obtained by subtracting the fitted levels at that level from the response
vector (and dividing by the estimated within-group standard error, if \texttt{type="pearson")}. The fitted
values at level $i$ are obtained by adding together the population fitted values (based only on the
fixed effects estimates) and the estimated contributions of the random effects to the fitted values at
grouping levels less or equal to $i$.
\end{description}

%================================================================== %
\newpage
\section{Influence Measures with \textit{influence.ME}}

%==================================================================================%
\begin{itemize}
\item \texttt{influence.ME} provides a collection of tools for
detecting influential cases in generalized mixed effects
models. 

\item It analyses models that were estimated using lme4. The
basic rationale behind identifying influential data is that
when iteratively single units are omitted from the data, models
based on these data should not produce substantially different
estimates. 

\item To standardize the assessment of how influential a
(single group of) observation(s) is, several measures of
influence are common practice, such as DFBETAS and Cook's Distance.

\item In addition, we provide a measure of percentage change of the fixed point
estimates and a simple procedure to detect changing levels of significance.
\end{itemize}
%=============================================================================================================%

\textbf{influence.ME} calculates measures of influence for mixed effects models estimated with the lme4 R package. The
basic rationale behind measuring influential cases is that when iteratively single units are omitted
from the data, models based on these data should not produce substantially different estimates. 

Calculating measures of influential data for an LME model requires the re-estimation
of this model for each set of potentially influential data separately. The \texttt{estex(} function does this,
and returns the altered estimates resulting from each re-estimation. 

The main function in the influence.ME package is the \texttt{influence()}.

\begin{quote} Based on a priorly estimated
mixed effects regression model (estimated using lme4), the \texttt{influence()} function iteratively modifies
the mixed effects model to neutralize the effect a grouped set of data has on the parameters, and
which returns returns the fixed parameters of these iteratively modified models. These are used to
compute measures of influential data. (Nieuwenhuis et al, 2014)
\end{quote}

\subsection*{Using the influence.ME package}
Influence Analysis can onnly be carried out with LME models fitted using the functions in the \textbf{lme4} package. Such models are known as \texttt{mer} objects.
Hence the \texttt{estex()} function only works on LME
models of class \texttt{mer}.
The package developers advise that it is required that the \texttt{mer} model was estimated using a factor variable to indicate group levels.
When using something similar to \texttt{+ (1 | as.factor(variable))}, the function is not able of
identifying the correct grouping factors, and returns an error.

Executing this procedure can be computationally highly demanding, because \texttt{estex()} entails the re-estimation of the provided mixed effects model for each level of the specified grouping factor (after alteration of the data).


The \texttt{estex()} function requires the specification of two parameters: 
\begin{enumerate}
\item a mixed effects model is to be specified,
\item the grouping factor of which the influence of the nested observations are to be evaluated. 
\end{enumerate}

\subsection*{ The school23 Example }
In a published tutorial, Nieuwenhuis et al provide an example using a data set that is provided with the \textbf{influence.ME} \texttt{R} package. The \textbf{school23} data contains information on students’ performance on a math test, as well as several
explanatory variables. These data are subset of the NELS-88 data (National Education Longitudinal
Study of 1988). Both a selected number of variables and a selected number of observations are given.


The \textit{school23} data contains information on a math test performance of 519 students, who are nested within 23 schools. For this example,  analysts will be interested in the relationship between class structure (in this data measured at the school level) and students’ performance on a math test. The research question is: \textit{To what extend does the classroom structure determine the students’ math test outcomes?}

Initially, we will estimate the effect of class structure on the result of the math performance test, without any further covariates. We do take into account the nesting structure of the data, however, and allow the intercept to be random over schools. This model is estimated using the following syntax, and is assigned to an object we call ‘model’.
\begin{framed}
\begin{verbatim}
model <- lmer(math ~  structure + (1 | school.ID), data=school23)
summary(model)
\end{verbatim}
\end{framed}
The call for a \texttt{summary} of the model results in the output shown below. In this summary, the original model formula is shown, as well as the data on which this model was estimated. Both random and fixed effects are summarized. The amount of intercept variance associated with the nesting structure of students within schools is considerably large (23.8 compared with 81.2 + 23.8 = 104 in total). The effect of interest is that of the structure variable, which is -2.343 and statistically insignificant by most reasonable standards (t=-1.609).

\begin{framed}
\begin{verbatim}
Linear mixed model fit by REML 
Formula: math ~ structure + (1 | school.ID) 
   Data: school23 
  AIC  BIC logLik deviance REMLdev
 3802 3819  -1897     3798    3794
 
Random effects:
 Groups    Name        Variance Std.Dev.
 school.ID (Intercept) 23.884   4.8871  
 Residual              81.270   9.0150  
Number of obs: 519, groups: school.ID, 23

Fixed effects:
            Estimate Std. Error t value
(Intercept)   60.002      5.853  10.252
structure     -2.343      1.456  -1.609

Correlation of Fixed Effects:
          (Intr)
structure -0.982
\end{verbatim}
\end{framed}

In the syntax example below, the original object 'model' is specified, and 'school.ID' is the relevant grouping factor. school.ID is the name of the variable used to indicate the grouping factor when the original model was specified. The estex() function works perfectly when more than a single grouping is present in the model, but only one grouping factor can be addressed at once.

\begin{verbatim}
data(school23)
model <- lmer(math ~ structure + SES + (1 | school.ID), data=school23)
alt.est <- influence(model, group="school.ID")
cooks.distance(alt.est)
\end{verbatim}

%============================================================================= %
\subsection{Functionality of the influence.ME pacakge}
The R package influence.ME allows for the calculation measures of influential data for mixed effects models generated by lme4. To standardize the assessment of how influential an observation (or group of observations)is, several commonly encountered measures
of influence are used by \textbf{influence.ME}.


\begin{itemize}
\item DFBETAS is a standardized measure of the absolute difference
between the estimate with a particular case included and the estimate without that particular
case. 
\item Cook’s distance provides an overall measurement of the change in all parameter
estimates, or a selection thereof.
\end{itemize}

The \texttt{estex()} command computes revised estimates can subsequently
be entered to the \texttt{cooks.distance} and \texttt{dfbetas} commands, to calculate Cook’s Distance
and the DFBETAS (standardized difference of the beta) measures.

\subsubsection*{Example}

\begin{framed}
\begin{verbatim}
library(lme4)
model <- lmer(mpg ~ disp + (1 | cyl), mtcars)

#The function influence is the 
# basis for all further steps:

library(influence.ME)
infl <- influence(model, obs = TRUE)

# Calculate Cook's distance:
cooks.distance(infl)

# Can Plot Cook's distance:

plot(infl, which = "cook")
\end{verbatim}
\end{framed}

%================================================================================%



%--------------------------------------------------------------%
\subsubsection*{The \texttt{pchange} command}

The \texttt{pchange} command computes the percentile change, as a measure of influential data. This unstandardized measure can
serve to help interpret the magnitude of the influence single or combined grouping levels exert on
mixed effects models. 

The percentage change in parameter estimates between an LME model based on a full set of data, and a model from which a (potentially influential)
subset of data is removed. A value of percentage change is calculated for each parameter in the
model separately, based on the information returned by the \texttt{estex()} function.

\subsection*{sigtest}

The \texttt{sigTest()} function can test for changes in the level of statistical significance resulting from
the deletion of potentially influential observations


\subsubsection*{The \texttt{plot.estex} command}
This is a wrapper function to the \texttt{dotplot()} function in the \textbf{lattice} \textbf{R} package.
%=====================================================================================================%

\newpage
%====================%
% Diagnostics with nlmeU

\section{Leave-One-Out Diagnostics with \texttt{lmeU}}
Galecki et al provide a brief the matter of LME influence diagnostics in their book.

The command \texttt{lmeU} fits a model with a particular subject removed. The identifier of the subject to be removed is passed as the only argument

A plot ofthe per-observation diagnostics individual subject log-likelihood contributions can be rendered.

\subsection{Cook's Distance}
\subsection{Likelihood Displacement}

\end{document}
