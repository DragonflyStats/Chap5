\section{Case Deletion Diagnostics for LME models}

\citet{HaslettDillane} remark that linear mixed effects models
didn't experience a corresponding growth in the use of deletion
diagnostics, adding that \citet{McCullSearle} makes no mention of
diagnostics whatsoever.

\citet{Christensen} describes three propositions that are required
for efficient case-deletion in LME models. The first proposition
decribes how to efficiently update $V$ when the $i$th element is
deleted.
\begin{equation}
V_{[i]}^{-1} = \Lambda_{[i]} - \frac{\lambda
\lambda\prime}{\nu^{}ii}
\end{equation}


The second of christensen's propostions is the following set of
equations, which are variants of the Sherman Wood bury updating
formula.
\begin{eqnarray}
X'_{[i]}V_{[i]}^{-1}X_{[i]} &=& X' V^{-1}X -
\frac{\hat{x}_{i}\hat{x}'_{i}}{s_{i}}\\
(X'_{[i]}V_{[i]}^{-1}X_{[i]})^{-1} &=& (X' V^{-1}X)^{-1} +
\frac{(X' V^{-1}X)^{-1}\hat{x}_{i}\hat{x}' _{i}
(X' V^{-1}X)^{-1}}{s_{i}- \bar{h}_{i}}\\
X'_{[i]}V_{[i]}^{-1}Y_{[i]} &=& X\prime V^{-1}Y -
\frac{\hat{x}_{i}\hat{y}' _{i}}{s_{i}}
\end{eqnarray}








In LME models, fitted by either ML or REML, an important overall
influence measure is the likelihood distance \citep{cook82}. The
procedure requires the calculation of the full data estimates
$\hat{\psi}$ and estimates based on the reduced data set
$\hat{\psi}_{(U)}$. The likelihood distance is given by
determining


\begin{eqnarray}
LD_{(U)} &=& 2\{l(\hat{\psi}) - l( \hat{\psi}_{(U)}) \}\\
RLD_{(U)} &=& 2\{l_{R}(\hat{\psi}) - l_{R}(\hat{\psi}_{(U)})\}
\end{eqnarray}


\addcontentsline{toc}{section}{Bibliography}

\bibliography{transferbib}
\end{document}
