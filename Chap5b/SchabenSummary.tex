\documentclass[00-ResidualsMain.tex]{subfiles}
\begin{document}
	
	% 1.  Cook's Distance 1977
	% 2.  Extension to 
	% 3.  Demidenko 
	% 4.  Schabenburger
	%---------------------------------------------------------------% 
	
	\textit{schabenberger} examines the use and implementation of
	influence measures in LME models.
	
	Influence is understood to be the ability of a single or multiple
	data points, through their presences or absence in the data, to
	alter important aspects of the analysis, yield qualitatively
	different inferences, or violate assumptions of the statistical
	model (\textit{schabenberger}).
	
	Outliers are the most noteworthy data points in an analysis, and
	an objective of influence analysis is how influential they are,
	and the manner in which they are influential.
	
	\textit{schabenberger} describes a simple procedure for quantifying
	influence. Firstly a model should be fitted to the data, and
	estimates of the parameters should be obtained. The second step is
	that either single of multiple data points, specifically outliers,
	should be omitted from the analysis, with the original parameter
	estimates being updated. This is known as `leave one out \ leave k
	out' analysis. The final step of the procedure is comparing the
	sets of estimates computed from the entire and reduced data sets
	to determine whether the absence of observations changed the
	analysis.
	
	
	
	A residual is the difference between an observed quantity and its
	estimated or predicted value. In LME models, there are two types
	of residuals, marginal residuals and conditional residuals. A
	marginal residual is the difference between the observed data and
	the estimated marginal mean. A conditional residual is the
	difference between the observed data and the predicted value of
	the observation. In a model without random effects, both sets of
	residuals coincide.
	
	\textit{schabenberger} notes that it is not always possible to
	derive influence statistics necessary for comparing full- and
	reduced-data parameter estimates. 
	
	
	\begin{abstract}
		\noindent This paper reviews the use of diagnostic measures for LME models in SAS. This text has been widely cited by texts that don't deal with SAS implementations.
	\end{abstract}
	
	
	%---------------------------------------------------------------%
	\section*{Schabenberger: Summary and Conclusions}
	\begin{itemize}
		\item Standard residual and influence diagnostics for linear models can be extended to linear mixed models. The dependence of fixed-effects solutions on the covariance parameter estimates has important ramifications in perturbation analysis. 
		\item To gauge the full impact of a set of observations on the analysis, covariance parameters need to be updated, which requires refitting of the model. 
		\item The experimental INFLUENCE option of the MODEL statement in the MIXED procedure (SAS 9.1) enables you to perform iterative and noniterative influence analysis for individual observations and sets of observations.
		
		\item The conditional (subject-specific) and marginal (population-averaged) formulations in the linear mixed model enable you to consider conditional residuals that use the estimated BLUPs of the random effects, and marginal residuals which are deviations from the overall mean. 
		\item Residuals using the BLUPs are useful to diagnose whether the random effects components in the model are specified correctly, marginal residuals are useful to diagnose the fixed-effects components. 
		\item Both types of residuals are available in SAS 9.1 as an experimental option of the MODEL statement in the MIXED procedure.
		
		\item It is important to note that influence analyses are performed under the assumption that the chosen model is correct. Changing the model structure can alter the conclusions. Many other variance models have been fit to the data presented in the repeated measures example. You need to see the conclusions about which model component is affected in light of the model being fit.
		\item  For example, modeling these data with a random intercept and random slope for each child or an unstructured covariance matrix will affect your conclusions about which children are influential on the analysis and how this influence manifests itself.
	\end{itemize}
	
\end{document}
