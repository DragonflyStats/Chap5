\documentclass[00-ResidualsMain.tex]{subfiles}
\begin{document}
	
	\newpage
	%-----------------------------------------------------------------%
	\section*{Diagnostic Methods for OLS models}
	% Cook's Distance for OLS models
	% http://www.amstat.org/meetings/jsm/2012/onlineprogram/AbstractDetails.cfm?abstractid=305411
	Influence diagnostics are formal techniques allowing for the identification of observations that exert substantial 
	influence on the estimates of fixed effects and variance covariance parameters. 
	
	The idea of influence diagnostics for a given observation is to quantify the effect of omission of this observation 
	from the data on the results of the model fit. To this aim, the concept of likelihood displacement is used. 
	
	%---------------------------------------------------------------%
	% We have developed a function in R, which allows performing influence diagnostics for linear mixed effects models 
	% fitted using the lme() function from the nlme package. 
	% The use of the new function is illustrated using data from a randomized clinical trial.
	
	%---------------------------------------------------------------%
	
	\subsection*{Influence Diagnostics: Basic Idea and Statistics} %1.1.2
	%http://support.sas.com/documentation/cdl/en/statug/63033/HTML/default/viewer.htm#statug_mixed_sect024.htm
	
	The general idea of quantifying the influence of one or more observations relies on computing parameter estimates based on all data points, removing the cases in question from the data, refitting the model, and computing statistics based on the change between full-data and reduced-data estimation. 
	
	\newpage
	\section{Case Deletion Diagnostics} %1.6
	
	\textbf{CPJ} develops \index{case deletion diagnostics} case deletion diagnostics, in particular the equivalent of \index{Cook's distance} Cook's distance, for diagnosing influential observations when estimating the fixed effect parameters and variance components.
	
	\subsection{Deletion Diagnostics}
	
	Since the pioneering work of Cook in 1977, deletion measures have been applied to many statistical models for identifying influential observations.
	
	Deletion diagnostics provide a means of assessing the influence of an observation (or groups of observations) on inference on the estimated parameters of LME models.
	
	Data from single individuals, or a small group of subjects may influence non-linear mixed effects model selection. Diagnostics routinely applied in model building may identify such individuals, but these methods are not specifically designed for that purpose and are, therefore, not optimal. We describe two likelihood-based diagnostics for identifying individuals that can influence the choice between two competing models.
	
	Case-deletion diagnostics provide a useful tool for identifying influential observations and outliers.
	
	The computation of case deletion diagnostics in the classical model is made simple by the fact that estimates of $\beta$ and $\sigma^2$, which exclude the ith observation, can be computed without re-fitting the model. Such update formulas are available in the mixed model only if you assume that the covariance parameters are not affected by the removal of the observation in question. This is rarely a reasonable assumption.
	
	\section{Effects on fitted and predicted values}
	\begin{equation}
	\hat{e_{i}}_{(U)} = y_{i} - x\hat{\beta}_{(U)}
	\end{equation}
	
\end{document}
