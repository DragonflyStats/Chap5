\documentclass[Chap5amain.tex]{subfiles}
\begin{document}

% http://www.jstor.org/discover/10.2307/1269550?uid=3738232&uid=2&uid=4&sid=21103552726783

 % Abstract for CPJ paper
 % Mixed linear models arise in many areas of application. 
 % Standard estimation methods for mixed models are sensitive to bizarre observations. 
 % Such influential observations can completely distort an analysis and lead to inappropriate actions and conclusions. 
 % We develop case-deletion diagnostics for detecting influential observations in mixed linear models. 
 % Diagnostics for both fixed effects and variance components are proposed. 
 % Computational formulas are given that make the procedures feasible. 
 % The methods are illustrated using examples.

%------------------------------------------------------------------------------------------------------%

\subsection{Case Deletion Diagnostics for Mixed Models}


\citet{Christiansen} notes the case deletion diagnostics techniques have not been applied to linear mixed effects 
models and seeks to develop methodologies in that respect.


\citet{Christiansen} develops these techniques in the context of REML

%---------------------------------------------------------------------------%
\newpage
\section{Matrix Notation for Case Deletion} %1.14


\subsection{Case deletion notation} %1.14.1


For notational simplicity, $\boldsymbol{A}(i)$ denotes an $n \times m$ matrix $\boldsymbol{A}$ with the $i$-th row
removed, $a_i$ denotes the $i$-th row of $\boldsymbol{A}$, and $a_{ij}$ denotes the $(i, j)-$th element of $\boldsymbol{A}$.


\subsection{Partitioning Matrices} %1.14.2
Without loss of generality, matrices can be partitioned as if the $i-$th omitted observation is the first row; i.e. $i=1$.


\end{document}