	\documentclass[Main.tex]{subfiles}
	\begin{document}
		
		%---------------------------------------------------------------------------%
		\newpage
		\section{Iterative and non-iterative influence analysis} %1.13
		\citet{schabenberger} highlights some of the issue regarding implementing mixed model diagnostics.
		
		% A measure of total influence requires updates of all model parameters.
		% however, this doesnt increase the procedures execution time by the same degree.
		
		\subsection{Iterative Influence Analysis}
		
		%----schabenberger page 8
		For linear models, the implementation of influence analysis is straightforward.
		However, for LME models, the process is more complex. Update formulas for the fixed effects are available only when the covariance parameters are assumed to be known. A measure of total influence requires updates of all model parameters.
		This can only be achieved in general is by omitting observations, then refitting the model.
		
		\citet{schabenberger} describes the choice between \index{iterative influence analysis} iterative influence analysis and \index{non-iterative influence analysis} non-iterative influence analysis.
		
		
		
		%---------------------------------------------------------------------------%
		\newpage
		\section{Iterative and non-iterative influence analysis} %1.13
		\citet{schabenberger} highlights some of the issue regarding implementing mixed model diagnostics.
		A measure of total influence requires updates of all model parameters.
		
		however, this doesnt increase the procedures execution time by the same degree.
	\subsection{Iterative Influence Analysis}
	
	%----schabenberger page 8
	For linear models, the implementation of influence analysis is straightforward.
	However, for LME models, the process is more complex. Update formulas for the fixed effects are available only when the covariance parameters are assumed to be known. A measure of total influence requires updates of all model parameters.
	This can only be achieved in general is by omitting observations, then refitting the model.
	
	\citet{schabenberger} describes the choice between \index{iterative influence analysis} iterative influence analysis and \index{non-iterative influence analysis} non-iterative influence analysis.
	
	
	
	\subsection{Iterative vs Non-Iterative Influence Analysis}
	%\subsection{ITERATIVE VS. NONITERATIVE INFLUENCE ANALYSIS}
	While the basic idea of influence analysis is straightforward, the implementation in mixed models can be
	tricky. For example, update formulas for the fixed effects are available only when the covariance parameters
	are assumed to be known. At most the profiled residual variance can be updated without refitting the model.
	
	A measure of total influence requires updates of all model parameters, and the only way that this can be
	achieved in general is by removing the observations in question and refitting the model. 
	
	Because this “\textbf{bruteforce}”
	method involves iterative reestimation of the covariance parameters, it is termed \textbf{\textit{iterative influence
	analysis}}. Reliance on closed-form update formulas for the fixed effects without updating the (un-profiled)
	covariance parameters is termed a noniterative influence analysis.
	
	An iterative analysis seems like a costly, computationally intensive enterprise. If you compute iterative
	influence diagnostics for all n observations, then a total of $n + 1$ mixed models are fit iteratively. This does
	not imply, of course, that the procedure’s execution time increases n-fold. Keep in mind that
	\begin{itemize}
		\item iterative reestimation always starts at the converged full-data estimates. If a data point is not influential,
		then its removal will have little effect on the objective function and parameter estimates. Within
		one or two iterations, the process should arrive at the reduced-data estimates.
		\item if complete reestimation does require many iterations, then this is important information in itself. The
		likelihood surface has probably changed drastically, and the reduced-data estimates are moving away
	\end{itemize}
	from the full-data estimates.
	
	\end{document}
